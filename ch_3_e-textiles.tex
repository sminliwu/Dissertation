\chapter{Naming E-textiles}
\label{ch_e-textiles}

Just like how I unpacked what ``craft" means in the introduction chapter, interrogating the term ``e-textiles'' also uncovers issues of how technology relates to other things in the world. What exactly \textit{are} ``e-textiles''? I had briefly explained e-textiles as ``the integration of electronics and textiles'', but this is less of a definition of specific features, and more of a phrase to broadly capture the scope of e-textiles. \revision{Language is one of the classic examples of coproduction in STS \cite{jasanoff_states_2010}, as how we describe and name something not only shapes how we perceive it, but also its place in constructed reality.}

The term we use to describe this technology is also a way to give an \textit{identity} to an emergent field. Like the identity of a human individual, the nebulous identity of a community or discipline (if e-textiles can even be called that) can be examined for the experiences that shaped it and connections to broader sociopolitical contexts. \revision{And if we can more clearly grasp this context, the more intentionally we may respond to socio-political-technical factors to shape the technology being designed. Of course, I am framing this discussion in a very anthropocentric way. But I think I can assume that the reader of this dissertation is a fellow human like myself; and thus, this language discussion is an introspection on how we humans construct the more-than-human agents in the domain.}

\section{Are ``E-Textiles'' and ``Smart Textiles'' Different?}
\label{sct_e-textiles_motivation}

I noticed the conflicting language for e-textiles when I first began my PhD. I had first heard about the field from an arts perspective, learning about communities and events such as the E-Textiles Summer Camp \cite{noauthor_httpetextile-summercamporg_nodate} and E-Textiles Spring Break which focused on e-textiles as an artistic practice. When I arrived in Boulder as a student in the college of engineering, I quickly noticed that the term ``smart textiles'' was often used to refer to the same technologies that ``e-textiles'' encompassed. When preparing the AdaCAD paper with Mikhaila and Laura (see \ref{ch_adacad}), I observed how researchers whose work straddled the interface of arts and engineering would use the two terms interchangeably. As someone who grew up in an immigrant community, I instinctively recognized this as code-switching.

\revision{\citedterm{Code-switching} refers to when a multilingual person switches between different languages. \cite{alexiadou_grammar_2016} This linguistic tactic can serve several purposes, such as the person needing to adapt to different social contexts, or when one language's vocabulary is just more evocative than the other's. While the term most often describes switching between human languages like English and Mandarin Chinese (my personal example), code-switching also applies across differences in domain-specific terminology, e.g. how a handweaver and a textile mill engineer may describe woven cloth in different terms.}

During the summer of 2020, I collaborated with Madison Maxey and LOOMIA Electronics, a start-up for e-textiles prototyping components, to turn this question into a larger investigation of what makers wanted from the technology. The original question of language difference was interesting to start with, as I was intrigued by the potential cultural tensions that it might reveal, while LOOMIA would have a more immediately-applicable insight into what language their target market might prefer. We formulated a language survey for e-textiles practitioners as the first phase of the collaboration, which was sent to the company's mailing list.
Following the survey, we planned a second and third phase of the social research study, shifting from casting a wide net through a survey to more targeted, personal interviews.

In this chapter, we will review the overall study design, but only discuss that first language survey phase. Towards the end of the dissertation in Chapter \ref{ch_speculations}, we'll come back to the interview phases. However, I will introduce concepts and methods here that may not have appeared until the later study segments, as retroactively, they clarified the survey findings.

\section{Study Design: Three Aspects of E-textiles Practice}

As e-textiles practitioners ourselves studying practices in our own network, this study was designed for the authors' two interrelated purposes: to inform product development for a growing e-textiles market, and to survey implicit design values within the discipline which might impact manufacturability and sustainability. Both goals concern the future of e-textiles as it ``scales" as a technology and social practice. While the study's data collection did not specifically target sustainability as a design value, the research question that emerged to guide our analysis and reflections can be summarized as follows:

% indent
\begin{quote}

\texttt{E-Textiles Research Question:}

How can sustainability manifest in material, implicit ways for e-textiles practitioners, including those whose careers do not explicitly focus on sustainability and those who are not necessarily researchers on the topic? What could \textit{doing sustainability} look like for the future of e-textiles?
\end{quote}  

We will describe these methods in a chronological narrative of our research process. To preserve confidentiality (especially in ensuring honest product feedback), I anonymized all participants' data as the principal investigator and assigned pseudonyms for dissemination to my collaborators.
% \begin{figure}[h]
%     \centering
%     \includegraphics[width=2.5in]{figures/Assembling Seams FIGURES-Study Design.png}
%     \caption{A study design targeting three distinct components of e-textiles practice: language, prototyping, and manufacturing. The graphic shows the components as completely separate entities merely for clarity. The components represent distinct, but possibly overlapping and entangled aspects which e-textiles practitioners encounter.}
%     \label{fig:study-design}
% \end{figure}
Our study was divided into three segments, each targeting a component of e-textiles practice to investigate for existing sustainability dialogue and thinking, along with potential development. These segments were:

\begin{enumerate}
  \item \textbf{Language} used by practitioners to describe the e-textiles domain.
  \item \textbf{Prototyping} practices for future e-textiles technologies. 
  \item \textbf{Manufacturing} perspectives on scaling future e-textiles products.
\end{enumerate}

The procedures for each study segment relied on sampling within our existing networks, and my own personal perspective heavily influenced how I conducted the research. As such, we recognize that
% We acknowledge that, by drawing on design justice and sampling within our existing networks, 
our methods situate our collected data and analyses within our own subjectivities as researchers. As such, our \keyterm{speculative construction} as developed through the findings and discussion is limited to our specific e-textiles community of practice, rather than the multifaceted discipline at large.
% However, as our research questions sought possibilities rather than generalizable

\section{The Language Survey}

The language segment of our study was designed to probe how e-textiles could identify as a field (i.e. ``e-textiles" vs. ``smart textiles") and to see if differences in the choice of these terms suggested particular sustainable values.
We chose to conduct an \textit{online survey} as our main data collection method in this segment to obtain a `wide-angle lens` of a domain that may be under-studied or new \cite{braun_online_2020}.  
In cultural anthropology, surveys can be used to quickly establish connections with a participant community \cite{bernard_handbook_2014}, which is similar to ``agile" user experience (UX) design's use of surveys to quickly obtain market research or audience feedback while simultaneously building a customer base \cite{buley_user_2013}. 

\subsection{Design}

Our e-textiles language survey was particularly inspired by the popularity of a 2013 language survey
in the New York Times (NYT) \cite{katz_how_2013} that targeted USA regional dialects. Not only was the ``dialect quiz" the most popular content that NYT published that year, but the quiz sparked lively discussion on- and off-line among Americans who were not previously aware of dialect differences, bringing an analysis of subtle cultural differences to a general audience. Our language survey borrowed elements of the quiz by first asking participants brief questions, answerable with one or a few clicks, on language preferences between the following set of terms:
% like ``e-textiles'', ``smart textiles'', ``functional fabrics'', or ``stretchable electronics''.

\begin{quote} \vspace{-2em}
  \singlespacing
  \texttt{
    \begin{itemize}
      \item E-textiles
      \item Smart textiles
      \item Functional fabrics
      \item Soft circuits
      \item Flexible circuits/devices
      \item FHE (Flexible Hybrid Electronics)
      \item Stretchable electronics
    \end{itemize}  
  }
\end{quote}


We selected these terms from literature reviews and industry communications which the authors had encountered, with terms being trimmed from the list for redundancy if they already shared some combination of keywords (e.g. ``smart fabrics" was not included). We were careful not to include any terms that spoke to more specific integrations within the e-textiles domain, such as ``smart garments" which only refer to certain on-body applications of e-textiles.
% In this same section of the survey, we asked participants to self-identify the follow demographic information:
% \begin{itemize}
% \item age bracket (10 year intervals)
% \item gender (in a text box)
% )
% \item number of years in e-textiles
% \item current job title
% \item best category for their role (choices: engineer, creative technologist, artist/craftsperson, and ``other" write-in option)
% \end{itemize}
% The age and gender fields were formatted to be inclusive, yet compatible with coding during analysis \cite{fonseca_designing_2020, couper_web_2001}. As an aside, we decided not to include race in the questions because % REASONS
% A1 and A2, as BIPOC (Black/Indigenous/People of Color) technologists working in mixed spaces where discussing issues of diversity, equity, and inclusion was not guaranteed as welcome, were personally not ready to navigate the topic in an industry context.

After this first section of quick and intuitive responses, the survey participants had an option to continue onto more questions like asking respondents to explain any language differences they wanted to qualify from the first set of questions, and what (if any) differences they saw between e-textiles and ``flexible" or ``wearable" device development. Additionally, we prompted participants to share ideas they held about the ideal qualities of future e-textiles items, as well as what e-textiles applications they hoped to see.

\subsection{Distribution and Initial Analysis}

The language survey was an anonymous Google form which was distributed most notably to LOOMIA's public mailing list. A2 noted that the company had over 10,000 subscribers on the list in May 2020, and sent content such as updates on the company's technology, new collaborations, and recent publicity.
% However, Flexsu had not tracked what portion of these subscribers were actively reading or engaging, so there is no way to gauge the response rate to the survey. Because we were using Flexsu's communication channels, where A2 more commonly used the term ``e-textiles", the survey title and follow-up communications also used this language.
We aggregated the multiple-choice responses and demographics data to create quantitative charts. Qualitatively, our analysis was done through open coding \cite{denzin_sage_2005}, using the text responses starting with the questions as initial structural codes (e.g. desired features and preferred terms) that could then be analyzed for values-based themes. 

\subsection{Community Discussion}

To supplement our analysis and following our NYT inspiration in provoking community discussion on language, we collected this preliminary analysis to present back to the respondents in a Zoom webinar. We called the event an ``E-Textiles Town Hall", making the presentation short and generally accessible, and devoting half of the 45 minute scheduled time for questions or comments from the attendees. We recorded and transcribed the event, coding it with the survey responses. % how did we code the transcripts

\section{Findings: Differing Value Alignments}

The complete dataset from the Language segment consisted of 65 survey responses, along with the audio recording, text chat logs, and anonymized transcript of the E-Textiles Town Hall. 
% % section intro/overview
% The language survey ultimately probed the question: who counts as an e-textiles developer? What does the e-textiles house encompass, and who's in here for us to talk to about sustainability? In analyzing what factors influenced different names for the field -- ``e-textile", ``smart textile", ``flexible hybrid electronics", etc. -- we asked: does sustainability influence conceptions of these terms? What are the features of an e-textiles professional identity?
The survey, along with the town hall discussion which we hosted, solicited a variety of opinions on what constituted e-textiles practice, as well as what were the desired future technologies from the field. As the walls and furnishings of a building shape the occupants' relationships with each room, the language used to describe e-textiles shaped and reflected the respondents' relationships with the technological domain.
Overall, the 65 participants showed thoughtful engagement with the survey and reflection on their own language use, considering that there was no material incentive for participation. In their text comments, several participants reacted to the questions with, ``Interesting!" or said that they ``really hope" for some particular change in how e-textiles was progressing. 
Over half of the respondents ($n=35$) chose to continue to the optional questions, and each person's total word count from the optional text response questions averaged 122 words, with the longest response exceeding 400 words. 
Among these signs of investment in the values of e-textiles, sustainability was only tangentially mentioned a handful of times, such as identifying ``recyclability" when asked about the desired features or qualities of future e-textiles products. Responses about personal comfort (``soft", ``not scratchy") and convenience (``washability"), as well as associations with consumer devices (``health trackers", ``Internet of Things") were overwhelmingly more common.

Importantly for investigating relationships and professional identity within e-textiles, the language survey let us scrutinize the main disagreement we have danced around this paper thus far: is it ``e-textile" or ``smart textile"? The answers were conflicted. Several participants especially noted a difference in how they perceived ``e-" versus ``smart" as a prefix. One associated ``e-" with ``e-waste", which seemed to contaminate the term ``e-textiles". They preferred ``smart textiles", as the prefix ``smart" was shared with phrases such as ``smart homes" and ``smart cars". Presumably, this person hoped for positive things from the e-textiles field and thus wanted a term for the field without negative associations. On a different ideological note, some participants argued that ``e-textiles" was better because ``smart" was too ambiguous of a label. One respondent tersely stated that ``every textile is smart in a way", so the term had an ``empty meaning". Another pointed to a similar selection of ``smart [device]" terms, making the label seem ``gimmicky" to them. 

Comments on the other terms -- ``functional fabrics", ``soft circuits", ``flexible [circuits/devices/hybrid electronics]" -- focused on how these terms suggested specific features or characteristics of things being built under the e-textiles banner. Descriptors such as ``flexible" and ``stretchable" only covered a single characteristic, which might not be the priority in all prototypes or products. However, even words based on tangible components of e-textiles were up for debate, such as ``circuits", ``fabric", or ``textile" as these all denote design paradigms, fabrication processes, and materials tied to specific technical domains such as electrical engineering or textiles manufacturing. We see that a proper name for the field must capture both the observable features of the desired technological developments, as well as the subjective hopes and speculative visions for the technology. As an emergent concept, e-textiles is already broadly defined, and such a proper name would also be broad. However, the term must also simultaneously be specific enough to set the field apart from others.

In giving opinions on what \textit{is} e-textiles and what is \textit{not}, we also asked participants on what they perceived to be the degree of overlap between e-textiles and ``wearable and flexible devices". One participant was seemed surprised at their lack of association between the concepts, answering, ``Not necessarily and I wonder why. I think there has been such a fashion push in the media that people aren’t thinking about the applications LOOMIA is interested in- like industrial workers, car seats..." The inability of some participants to unite some concepts with e-textiles, while still others were unable to clearly distinguish the field from related technologies, speaks to an undefined e-textiles group identity. 

However, one potential site of consensus was around the term ``creative technologist" for an e-textiles practitioner. In selecting the category of their career, 30.7\% of participants($n=20$) best described their practice as such, making it the most common response. The second-most common response was ``engineer" with 24.6\% of participants ($n=16$). Upon examining the creative technologists' responses to their specific job title, their roles ranged from ``start-up manager" to ``maker/programmer" to ``marketing director" to ``textile engineer", suggesting that the identity ``creative technologist" resonates with a wide variety of practices within e-textiles. Ultimately, these definitional tactics for drawing the boundaries of e-textiles influence the relationships of e-textiles practitioners with \textit{each other}. In understanding how language indicates and transmits e-textiles's disciplinary values, we can begin thinking of tactics that speak to both e-textiles and sustainability.

% \section{Methods: Crafting a Speculative Construction}

% Our methods incorporated elements of qualitative interviews \cite{braun_online_2020, fontana_interviewing_2007} and thematic analysis \cite{maguire_doing_2017, braun_using_2006} in social research to collect participant data.
% We use the term \keyterm{speculative construction} to describe our primary generative tactic for bridging these qualitative observations with ideas from design justice and speculative design to envision a sustainable e-textiles community of practice. Foremost, our vision centers \citedterm{design justice} \cite{costanza-chock_design_2020} and finds roots in observed practices and politics of e-textiles. The ``speculative" component alludes to \citedterm{speculative design}, notably summarized in Dunne \& Raby's \textit{Speculative Everything} \cite{dunne_speculative_2013} and often overlapping with \textit{design fiction}, which has been taken up in HCI to push design beyond specific objects to consider designed artifacts as ideas that suggest possible futures and systems, as well as alternative presents and histories \cite{akama_speculative_2016, baumann_infrastructures_2017, wakkary_sustainable_2013}. However, we specifically describe this work as \textit{speculative} rather than a full \textit{speculation}, as we do not present a coherent design world but still draw on the methodology's core concepts.

% One such core value in speculative design is considering the wide array of \textit{possible} futures and identifying the \textit{plausible} futures for deeper speculation. Out of these plausible ones, researchers can inquire into \textit{preferable} futures aligned with certain values. We use our data to guide our viewfinder to ``plausible" futures, and design justice to focus on ``preferable" futures.
% Prior research has also taken up speculative methods to center sustainable values. Especially relevant to our study of sustainability in emergent technology is Liu et al.'s exploration of ``collaborative survival" \cite{liu_design_2018}, stemming from a speculative design inquiry in pushing sociotechnical entanglements towards more ``preferable" sustainable futures, as well as Wong et al.'s toolset for ``infrastructural speculations" \cite{wong_infrastructural_2020}. Both of these works offered us tactics for inquiring into a context for an emergent technology such as e-textiles that considers the complex, often-fraught sociopolitical factors that make sustainability such a wicked problem. 

% Where design justice enriches speculative design's framework is at the originating point of %and Stuart Candy 
% the ``possibility cone" or ``futures cone" from Dunne \& Raby and Stuart Candy, located in the present. Applying a lens of intersectional justice shows us that there is not simply one singular way to experience the present due to multiple axes of systemic oppression, placing individuals at different coordinates along race, gender, class, etc. To complicate the present, and foreground social justice (of which sustainability is a critical part), Costanza-Chock's elaboration of \textit{design justice} \cite{costanza-chock_design_2020} describes a \citedterm{matrix of domination} that selects for more socially privileged design perspectives, separating design sites into the \textit{privileged} (e.g. Silicon Valley, university hackathons) and the \textit{subaltern} (e.g. auto shops, weaving studios). We believe that design justice magnifies the present point so that it no longer appears to be a singularity. Instead, it is now a debris field of alternate histories and possibilities truncated by past violence.
% , illustrated in Fig. \ref{fig:justice-futures-cone}. 
% Broadening the speculative cone allows us to look beyond success stories to uncover more humble points of pride, as well as frustration, and failure -- thus expanding our sense of possible futures. Furthermore, design justice's core strategy of \textit{retooling} -- designing tools that dismantle the established matrix of domination and construct a new, justice-oriented sociotechnical order -- offers a framework for designing these tools as components for building further speculation, hence the ``construction" descriptor for our analysis.

% can you say something here about what the difference between speculation and design recommendations? 


% sw - what I learned from making this graphic is that sometimes, making a graphic can help me write a section, even if the graphic doesn't make it into the final paper

% combine with submerged perspectives (Decolonial Perspectives) and subaltern design sites (Design Justice) -- adding in justice as a metric for ``preferability" of a future

Language is a notable example of coproduction: language is shaped by what it describes, and the realities described are in turn altered. The language survey was one of three collaborative research activities conducted that summer. However, we will save the other two for later to explore more tangible interventions of retooling.

