\chapter{Background}

\revision{In the previous chapter, I introduced and defined the key concepts we will deal with throughout this dissertation. Before diving into the specifics of my own research, we will first review existing approaches to craft and sustainability through retooling and coproduction. Many of these related works fall under the banner of human-computer interaction (HCI) and adjacent communities, e.g. ubiquitous computing (UbiComp). Many of these works also represent perspectives on craft from the arts, history, and science, technology, and society (STS) studies. These strands all converge within e-textiles research, and likewise, e-textiles work takes place in many of these disciplines. Rather than disciplinary boundaries, I want to examine e-textiles research in the context of three domains of making: digital fabrication, textiles, and social justice. Each of these domains contextualizes craft and sustainability in a unique way, thus animating different e-textiles design possibilities. To tie our discussion together, we will examine these domains for their common strands of ``retooling'' and ``coproduction''.}
% \todo{As this intersection of intersections falls across several domains, including human-computer interaction (HCI), textiles, and social justice; we will review these areas of research under the common lens of ``craft".}

\section{\revision{Digital Fabrication}}
\label{sect_bg_digital-fab}

\revision{\textbf{Digital fabrication (DF)} refers generally to fabrication machines and methods controlled by a computer, such as CNC milling, 3D printing, and laser-cutting \cite{mueller_digital_2016}. As the technology has matured and become more accessible to the general public, DF has become a key prototyping tool in areas such as architecture \cite{swaminathan_optistructures_2020}, healthcare \cite{hockaday_rapid_2012}, and fashion \cite{sarmakari_digital_2021,mcquillan_digital_2020}. Furthermore, it has become its own site of sociotechnical research in HCI as DF reconfigures relationships among humans, machines, data, and materials. In fact, digital fabrication has already had such a profound impact that some hail it as part of a ``Fourth Industrial Revolution'' \cite{schwab_fourth_2017}}. % HCI becomes increasingly nebulous and sprawling as researchers expand what constitutes ``interaction'' and what could be a ``computer''. Among the many discourses within HCI, we will focus primarily on design and fabrication. These making-oriented domains provide space for e-textiles research, as well as providing us with several neighbors who also tinker on unconventional, tangible computers.

% \subsection{\revision{A New Industrial Revolution}}

\subsection{\revision{Craft and Human-Machine Relations}}

HCI has developed a strong making-oriented sector with the rise of DF. Emerging practices such as ``computational craft" and ``hybrid craft" combine tangible materials with digital interfaces as their form of craft. E-textiles are often included with these crafts \cite{posch_integrating_2018,vasudevan_make_2015,petrelli_exploring_2016,buechley_crafting_2012}. In a 2018 paper,
Frankjaer \& Dalsgaard place these discourses in conversation with each other and with non-HCI craft scholarship to highlight how craft-based HCI research uniquely ``perceives and approaches the use of materials and techniques" \cite{frankjaer_understanding_2018}. Citing \revision{Richard Sennett's} three-part theorization of craft as ``questioning, localizing, and opening'' \revision{\cite{sennett_craftsman_2008}}, they discuss how artifacts of craft-based HCI research fall into this framework.

% provides examples of how retooling design impact practices around hardware and physical media. 
Some of these craft-based inquiries involve
% HCI researchers have created tools for 
combining 3D-printing with ceramic crafts as ``hybrid assemblages" \cite{efrat_hybrid_2016}, designing circuits on everyday materials such as paper and fabric \cite{kato_paper-woven_2022,jones_e-darning_2021,buechley_lilypad_2008}, as well as modifying a fabrication process to achieve unconventional interactions (e.g. machine-knitting flexible robots) \cite{albaugh_digital_2019}. 
% At the intersection of HCI, fabrication, UIST, and embedded hardware, 
\revision{Yet these very projects also partly refute Frankjaer \& Dalsgaard by defying their delineation of craft as a ``soft'' approach, in opposition to a ``hard'' approach (typically constructed as scientific or engineering-oriented). Unsurprisingly, Frankjaer's and Dalsgaard's framing is based on a theory of computational epistemology put forth by Turkle \& Papert \cite{turkle_epistemological_1990} that relies on the gender binary. If you know who I am, and how I identify, I think you can figure out why I would fundamentally disagree with these theoretical assumptions. 
% This theory, , is critical of masculine/feminine binaries and the conventional exclusion of girls/women from technical disciplines; however, it still assumes that there are two sides, one standing in opposition to another. 
The craft-oriented ``bricolage'' approach of synthesizing materials and techniques is supposedly in tension with a more calculating, machine-mediated ``technical'' analysis. Yet the craft-based fabrication machines and techniques which have emerged from HCI in recent years illustrate that, on the contrary, machines and human hands can work together quite nicely.}

We can see the impact of tools on the materials and techniques of a craft practice, often expanding the design possibilities and facilitating cross-domain knowledge exchange by establishing a ``grammar", such as the taxonomy presented in \cite{tran_oleary_grammar_2021}.
\revision{Further still, these projects suggest that digital fabrication, like many other crafts, is fundamentally entangled with its social context --- a coproduct of humans and the machines we create. In producing these new sociotechnical realities, DF indeed deserves a place in a new Industrial Revolution.} %Work that explicitly focuses on the social component of digital craft and fabrication include supporting craft practices in printing systems \todo{[136]} and a study of DIY modifications to tabletop equipment as shared on Twitter \cite{twigg-smith_tools_2021}. In considering the craft  %Clearly, \textit{retooling} in both a literal sense and in the design justice sense is an important part of craft.

\revision{I find some irony in calling DF both a craft and a revolutionary technology, given the historical tension between craft and industrialization. For instance, the British Industrial Revolution---dated roughly 1760--1830 CE, often called the ``first'' Industrial Revolution \cite{mokyr_british_2019}---centered on textiles innovations which mechanized and centralized production. Historians summarize the mechanization transition as a massive shift from ``hand-tool technology'' to ``machine-tool technology'' \cite{albrecht_what_2013}. Meanwhile, centralization not only re-organized workers and equipment, but also concentrated power in stratified, standardized management structures \cite{geraghty_technology_2003}, thus producing our modern notion of the ``factory''. These reforms also codified our models of industrialization and manufacturing. By removing humans from directly working with the materials (e.g. spinning yarn, weaving cloth), industrialization might be the antithesis to craft.}

\revision{In many ways, digital fabrication's revolution is an exact inversion of prior industrialization. Machines get smaller and leave centralized factories to return to distributed production sites. People use these tools at home or in local community spaces to manufacture things in small quantities. Rather than ``cottage industries'', we refer to the production sites as ``maker/hacker-spaces'' \cite{mueller_digital_2016,zhang_design_2018}. While some facets of society are more centralized than ever, such as wealth and data hosting infrastructure, digital fabrication reflects a desire to decentralize resources and production. If our current climate crisis is the result of past Industrial Revolutions and industrialization \cite{whyte_indigenous_2017}, then hopefully, this movement which holds such opposite goals will result in more sustainable alternatives.}

% Perhaps, decentralization in manufacturing recalls pre-industrial, craft-based economies. I argue that craft offers alternatives to industrialization's configuration of human-machine relationships Then hopefully, this rising interest in craft will   \cite{upadhyay_blockchain_2021}, the Internet of Things, circularity \cite{}}

% craft = decentralization
% but craft + DF offer alternative models for humans + machines 

\subsection{\revision{How Tools Matter}}

Research in HCI and other social dimensions of technological development have found that tool creation, as enabling research, has accelerated innovation in modern times -- such as the graphical user interface (GUI) in enabling more intuitive interactions with computers, ultimately enabling accessible personal computers, widespread Internet usage, and today's digital environments. HCI also recognizes that these tool inventions are themselves derived from earlier tools, inheriting characteristics and adapting features from existing systems \cite{hudson_concepts_2014}. This dynamic of continual adaptation and updating of ``new" tools from other work holds true for e-textiles design tools, which is why I frame my research and resulting contributions as products of retooling --- creating and modifying tools for a process. In fact, I argue that e-textiles practitioners must particularly attend to the retooling aspect when designing this technology, as the aforementioned ``existing systems" context for e-textiles encompasses thousands of years of human innovation in textile tools, from handheld tools to fully-mechanized knitting and weaving factory floors.

Creating and improving tools has driven much of humanity's technological progress -- tools let people make things taller, stronger, more precisely; envision more ambitious designs; even share their ideas across greater distances. That tools make certain tasks possible or easier, while constantly evolving, describes one process by which our technology and societal development are entangled. In engineering disciplines, this iterative process of ``retooling" mainly describes the literal reconfiguration and/or updating of tools for factories and other mechanized workflows \todo{[152]}. Yet, retooling has been particularly taken up by design justice as a framework for analyzing existing biases in technological design, the social inequities that our designs perpetuate, and how creating new tools or rehabilitating existing ones can ``retool" sociotechnical systems to advance justice for all \todo{[33]}---e.g. retooling social media content algorithms to address the racial, gendered, etc. biases they reinforce to support other forms of activism. 
% Sasha Costanza-Chock, in her book Design Justice, elaborates that retooling this particular technology will include developing ``intersectional user stories, testing approaches, training data, benchmarks, standards, validation processes, and impact assessments, among many other tools." Specific to my domain in e-textiles, I will define ``retooling e-textiles" as:

Particularly in HCI, where tools include software interfaces, electronic hardware, and social messaging platforms, tools have laid the foundation for interaction paradigms such as keyboard/mouse interfaces. As Mankoff and Hudson write about Technical HCI \revision{\cite{hudson_concepts_2014}}, a discipline which focuses on producing these tool inventions, there are two types of technical HCI inventions: direct and enabling. Direct research inventions create something that supports a long-term goal, like distance learning or accessibility for people who are blind and visually-impaired (BVI) or other end-user application. In contrast, enabling research does not directly address an end-user need, but rather enables other researchers to more easily address it. Enabling research includes tools, as well as systems and other inventions that improve basic capabilities for designers.

This framing of tools as ``enabling" longer term goals which can be explicitly values-driven, such as accessibility, educational equity, etc. implies that tools can have a profound impact on what values are foregrounded vs. diminished in a technological practice. Taking a long view of recent advances in HCI, we can see how the development of now-commonplace tools have shaped current design practices. In a review written by Myers et al. in 2000, on the ``Past, Present, and Future of User Interface Software Tools" \cite{myers_past_2000}, the authors take a retrospective on the two preceding decades of development in user interface software tools (UIST) which have enabled design disciplines such as computer-assisted design (CAD), user experience (UX) design, and human-centered design (HCD). They identify common themes in UIST which describe specific influences that design tools can have on their users' practices. Particularly relevant for e-textiles' emergent design space, where the nature and scope of practices is fundamentally yet-to-be-defined, are their themes ``threshold and ceiling", ``predictability", and ``moving targets". These terms all allude to the inherent uncertainty in designing a tool when its intended tasks are perpetually shifting, most directly by ``moving targets" that are present even in an established discipline. A design tool's ``threshold and ceiling" are the lower- and upper-bounds of the tool's capabilities, respectively, relative to a user's expertise; yet this requires that knowledge has been codified as relevant ``expertise". Aiming for ``predictability" in a tool implies knowing users' expectations for the task and that users even have expectations if the task is unknown. 

To explore their themes, Myers et al. give some key examples of ``successes" in UIST: scripting languages such as Python and Perl, and object-oriented programming that represents components of a virtual system to tangible design elements. Some ``failures" (i.e. a concept that failed to take hold) include systems based on ``automatic techniques" which would generate design features from high-level commands, which would theoretically lighten the human designer's burden of implementing low-level details. However, in reality, these systems often could only automate a limited set of design possibilities, suffered from being unpredictable for designers who could not understand the high-level grammar, and furthermore became increasingly unusable as designing user interactions expanded from ``desktop" devices to diverse form factors such as pagers and tablets. All in all, it seems that many of these design tools failed because they inhibited human agency in the process by obscuring fundamental aspects of the domain, such as machine specifics. These examples of ``failures" offer cautionary tales for toolmakers today, yet taken in conjunction with ``successes", we can also see how those failures left possible retooling paths unexplored, while our present-day tools may inherit the limitations of ``successful" predecessors.

By understanding these tools as fundamentally social and cultural objects, we also come to consider the political nature of their designs. As a senior scholar of technology and ethics, Langdon Winner explored the following question in a 1980 article: ``Do Artifacts Have Politics?" \cite{winner_artifacts_1980}, contending that technical designs indeed hold political stances and particularly, that infrastructure's technology could bias a society towards authoritarianism or democracy. Under this lens, technology is an instrument of enforcing political orders --- as a type of technology, design tools transmit political values through their ecosystem of practice. A broadly applicable example of HCI design values include ``seamlessness" and its alternative ``seamfulness" elaborated in \todo{[62]}. As the authors define the two: ``seamless design" emphasizes ``clarity, simplicity, ease of use, and consistency" in user interactions. ``Seamful design" emphasizes ``configurability, user appropriation, and revelation of complexity, ambiguity or inconsistency" which can create spaces for critical inquiries into broader social impacts. Both values speak to the designer's influence on downstream user agency, showing that tool design can influence values across a technology's social sphere of influence, from development to end-user application. 

% \subsection{Hybrids and Coproduction: Negotiating Textile Values in Technology}

Technical HCI and design justice, in their approaches to tools as influences on design practices and social values, would agree: tools matter. Together, with technical HCI's focus on the implementation and verification of retooling, and design justice's focus on the sociopolitical ramifications of retooling, these two facets of ``retooling" complement rather than contradict one another when applied to e-textiles. Retooling e-textiles could potentially involve any textiles tools across human history, many of which have historically had profound impacts --- the Jacquard loom and its role in the Industrial Revolution, for one. Some of this work envisions new manufacturing infrastructures for textiles that mimic the visions offered of additive manufacturing but focusing on soft goods. 
% Specifically, Pamela Liou, a designer and technologist, envisioned a new form of cottage industry supported through an open-source tabletop Jacquard loom called Doti [84]. This is mirrored in companies like WOVNS that focus on fabricating small runs of user designed products [154]. Along with other technologies like the Kniterate, we are beginning to see workflows where users can print textile products on demand [155]. 
Supporting this growth of small-scale textiles manufacturing hardware, new software protocols are being developed to develop fully shaped artifacts based on digital inputs \todo{[5,92]}. These tools also reflect the continuing textiles practice of modifying and updating one's tools for the craft, a practice far older than modern industrial production that is still visible among individual makers in contemporary crafting communities, e.g. a Facebook group ``Weaving Hacks" \todo{[46]} where members document how they modify their looms and re-appropriate everyday items.

% Returning to the specific case of e-textiles, Devendorf and Rosner give examples of how ``coproduction" animates different design concepts and possibilities from ``hybrid". For one, acknowledging how craft and computation are historically intertwined in coproduction yields an exploration of how hand-weaving core memory modules for the Apollo space program were an early e-textiles practice. In fact, a present- and future-centered metaphor such as ``hybrid" would also lose many other past instances of how textiles were crucial to the development of digital technology, as the authors of The Fabric of Interface [95] explore---textiles may in fact form the basis of design metaphors in our modern digital interfaces. Furthermore, contextualizing e-textiles as a coproduction creates space to consider the rich history and nonlinear reality of technological progress. We may find that the contemporary industrial focus on cotton textiles, globally, resulted from shifting away from a previous focus on linen textiles driven by European colonizers' desire for Indian cotton [13,135]. HCI practitioners today may find lessons from ``past" technology such as wooden Jacquard looms [47]. These examples highlight how we humans have collectively made trade-offs in our technology that were not necessarily ``optimal", which situates inquiries into technological futures.

% Much of the retooling in sustainable HCI to foreground environmental impact might also be seen as foregrounding human-nature coproduction. Under a broader theoretical framework of the Anthropocene, researchers have explored sustainable behavior by designing ways to bring users in greater physical contact with the environment [77,82,85] or approaches that question the fundamental orientation of HCI as on that is focused on ``ease of use" [82], making space for people to reflect and act with their environment as a coproducer, however challenging and uncomfortable it may be [35,81,86]. Specific to textiles and fashion, designers are generally aware of the unsustainability of the global industry and engaging with the aforementioned sustainable design strategies [23,75,140]. e-textiles practitioners have explored the unique affordances of textiles for sustainable design tactics, such as the ability to repair textiles-based technologies by darning [68] and inherent structural compatibility with designing for disassembly (which I explain in Section 2). Designers in e-textiles (and more broadly in textiles/fashion) recognize the particular social dimensions in which their artifacts are situated, so research on sustainable textiles often emphasizes personal relationships, intimate bodily contact, and sentimental value -- for example, Fletcher's Craft of Use [49] and Kuusk's work on service-based, personalized production of sustainable e-textiles [76]. Expanding beyond individual practices, sustainable textiles has found homes in distributed efforts coproduced between artists and industry entities, including the EU Wear Sustain Network [55], as well as textile waste marketplaces like Queen of Raw [156] and Fab Scrap [157].

\section{Textiles}

You might think, textiles are crafts, so what more is there to discuss? However, the relationship between craft and textiles is a fraught one. As textile production was at the center of the ``first'' Industrial Revolution in the 18th century, the textiles realm is one of the oldest battlegrounds for the craft and the not-craft. 

\subsection{Craft Techniques}

% \subsection{e-textiles: Integrating Materials, Structures, and Tools}
E-textiles seek to integrate electronic capabilities (sensing, actuating, wireless networking) with textile materials and structures. With the advent of ubiquitous computing and the Internet of Things driving the need for smaller, flexible, invisible electronic devices to enable digital connections at various scales \todo{[48]}, researchers have looked to textile practices for their vast knowledge of creating flexible and comfortable objects at scale. This research can take the form of weaving and knitting fabrics with embedded environmental sensors \todo{[1,73]}, spinning yarns that can be used as a battery or motor \todo{[87,118,144]}, and many more possible combinations. With various form factors, e-textiles promises applications in next-generation wearable devices for medical and athletic bio-monitoring \todo{[97,108]}, novel interactive garments for our everyday fashions \todo{[15,74,93]}, and even enabling greater degrees of ``smart" in smart homes and smart vehicles (e.g. flooring, car seating upholstery) \todo{[71,124]}. Textiles are everywhere in our built environment, so the possible avenues for electronic integration seem countless. Generally speaking, the landscape for e-textiles consists of integrating textiles and electronics within materials, structures, and design/fabrication tools.

% \subsubsection{Textiles Techniques}

The ``textiles" dimension of e-textiles spans the wide variety of textile design and fabrication techniques that humans have developed from (pre)history, each with their own configurations of materials, structures, and tools. Textiles can be categorized by levels of structural integration starting with raw materials, ranging from harvested cotton plants to a vat of liquid synthetic polymer. \todo{[45,60]} The first level of integration through some preparation or treatment is fiber (e.g. cotton, polyester), a disorderly bundle of filaments or ``fluff". The second level of integration twists or spins the fiber into yarn or thread, aligning the fibers in a long, often-multi-stranded (i.e. ``plied") larger filament. The next level of textile processes manipulate yarns to create cloth or fabric, such as knitting, weaving, knot-making, etc. Fabric serves as large sheets of material which can be arbitrarily shaped (e.g. by cutting, folding, draping) and assembled (e.g. by sewing) to cover a desired surface or form, creating complex garments and other applications. To cover one final level of integration, these textile assemblages composed of lower-level objects might undergo a finishing stage. For a summative example, look at a quilted winter coat. Multiple types of fabric are layered, along with a fiber stuffing, to create a water-repellant exterior that is also insulating and moisture-wicking on the interior, all sewn together with thread. To finish the coat, we also need non-textile attachments or findings such as zippers and may even add embellishments to the surface with extra textile elements like embroidery and patches.

This description of textiles as a multi-level technological system is a summary of knowledge that can be found in industry textbooks, as well as the lived experiences of textile craftspeople, who may not define their language so rigorously but are nevertheless experts in this technology. I intentionally construct textiles technologies in this manner to parallel how computing systems are understood through multiple layers of abstraction, from the physics of subatomic particles, to transistor logic, to integrated circuits and operating systems \todo{[57]}. My research practice focuses on woven e-textiles, a particular combination of materials, structures, and tools (yarn, woven cloth, loom). Yet this framing of a practice's materials, structures, and tools can be applied also to knitted e-textiles (yarn, knitted fabric, needles) and fundamental circuits in computing (silicon, transistors, circuit layout/printing). Working in woven e-textiles allows me to use the same language to describe textile design practices as well as digital system design, highlighting similar patterns of organizational logic that inspire novel computational challenges and design possibilities.

\subsection{Sustainability and History}

When we consider the histories of textile industrialization and its current legacy in driving globalized climate change and industrial waste, we realize that e-textiles is inheriting these legacies. As the practice is already engaging with existing textiles (and electronics) manufacturing structures when creating prototypes, e-textiles designers have a particular responsibility to attend to sustainable values, design, and development. 

Contemporary textile manufacturing builds upon centuries of iterative machine and infrastructural adjustments, accumulating the material practices, political trends, and design choices of the past. As an illustrative example of how this accumulated structure poses concrete challenges for creating more sustainable manufacturing systems, consider the history of linen versus cotton in consumer textiles. Today, textiles manufacturing is overwhelmingly dominated by cotton fibers, and years of tool development have created equipment especially for cotton (e.g. the cotton gin) and optimized processes that are universal to textiles (weaving, spinning, and finishing) to assume cotton fibers as the default. As a result, even though the industry is realizing cotton's harmful environmental impacts and has identified lower-impact fibers such as linen and hemp, the factories that have been optimized for cotton's short, airy fibers are ill-equipped to handle linen's long, smooth fibers. \cite{kozlowski_handbook_2020}

We use this example to illustrate how choices and values in tool design can propagate within infrastructure to have tangible consequences for (un)sustainability. This example becomes even more relevant to climate activism and intersectional justice when we consider how cotton was industrialized via European colonization of India, entangling its story with ``the story of the making and remaking of global capitalism and...the modern world.'' (from \textit{Empire of Cotton: A Global History} \cite{beckert_empire_2015}) We see opportunities for design interventions from e-textiles practitioners to take up values and develop tools that will push for an opposite, beneficial impact on sustainable development.

As e-textiles practitioners work to develop the technology to ``scale'' beyond lab prototypes into a future industry, we want to proactively raise questions of how e-textiles can develop as a \textit{sustainable} technology in its emergent stages. In pursuing sustainable e-textiles, we seek possible roles for HCI research to effectively serve this mission. What tactics could e-textiles draw upon from sustainable HCI, research through design, climate activism, and other bodies of work?

% In the following paper, we describe and reflect upon a collaborative study conducted between two academic researchers (A1 and A3) and one industry start-up CEO (A2, Flexsu company\footnote{pseudonym for double-blind review}) to investigate aspects of e-textiles practice (language, prototyping, and manufacturing), originally for implicit disciplinary values such as ``scalability'' and sustainability, that would both inform product development and open further opportunities for collaborative design inquiries. Our central finding of how social relationships drove developments in e-textiles design prompted connections to similar themes in environmental activism and sustainable futures, inspiring us to imagine how we might support relationship-building in e-textiles as a first step to create space for more changes relating to sustainability. We combined our observations with ideas from design justice and speculative design in what we call ``speculative constructions'', conceptualizing tools for us as e-textiles designers to build more sustainable communities of practice. We contribute two things: first, qualitative data to support our claim that relationship-building will play a central role in scaling e-textiles technologies; and second, our speculative constructions as provocations for other e-textiles designers to consider their sociocultural subjectivities (i.e. relationships) as design opportunities for engaging with sustainability and other broader implications of an emergent technology.   

\section{Social Justice}

While social justice efforts often deal with policy change, along with social awareness and visibility, they are ultimately concerned with the lived experiences of real people. Systemic oppression has viscerally tangible effects upon a person's body, home, food supply, and many other material aspects of their life. A craft-based lens on social justice, therefore, might frame efforts as ``making'' and ``unmaking'' the world in pursuit of a more just, equitable future.

\subsection{Craftivism}

The term ``craftivism'' is used today to refer to the activist strategy of hand-crafting an item to display/wear as protest, partake in civil disobedience, or otherwise participate in a political movement. Notably, one recent project is credited with codifying the current form of craftivism, the Pussyhat Project \cite{pussyhat_site,literat_crafting_2020}. Relying on social media platforms to grow, organize, and mobilize, the signature ``pussyhat'' -- a bright pink, usually-knit beanie with a crown shaped to look like cat ears -- came to symbolize feminist protest against the Trump presidency.

However, even before the Internet and global communication technologies, crafting has long been a mode of political resistance. From the 1987 AIDS quilt \cite{newmeyer_knit_2008} to Indigenous communities defying colonial erasure of their culture \cite{feder-nadoff_performing_2022,muskrat_magazine_indigenous_2013,flores_weaving_2021} to women covertly seizing social/economic power \cite{barber_womens_1996,parker_subversive_2010,smith_bauhaus_2014}, craft somehow has a power to subvert and question the status quo. These histories are another reason I embrace the language of ``retooling'' to describe societal transformation both metaphorically and literally.

\revision{The distributed, grassroots nature of craftivism recalls the decentralized spaces and equipment of digital fabrication from \ref{sect_bg_digital-fab}. Their similar alignments towards craft point to how activism and technical development might also share common tactics for realizing sustainability, despite operating in different settings. As an example of climate craftivism, I highlight the Fibershed movement \cite{fibershed}. The mission of the Fibershed organization is to create ``regional fiber systems'' that encompass textiles production from soil and sheep to a garment's end-of-life, advancing sustainability by implementing an alternative, equitable economy. Compare Fibershed's approach to ongoing research in sustainable fabrication and manufacturing: leveraging decentralized blockchain technologies \cite{upadhyay_blockchain_2021}, creating biodegradable materials \cite{vasquez_myco-accessories_2019}, developing workflows for reuse and remanufacturing \cite{nasr_fundamentals_2020,prendeville_circular_2017, 10.1145/3393914.3395894,wall_scrappy_2021}, to name a few. These works aspire towards a circular economy, so like Fibershed, implement features of their sustainable future. In my experiences in both social justice and technologist spaces, you must personally make and embody the future you want.}

\subsection{Making Futures and Worlds}

% Design justice expands the notion of ``values'' in design, to causes which unite several values into a desired social idea. 
Design justice's notion of ``retooling" foregrounds the power inherent in creating tools by emphasizing that tools create (and destroy) our social realities. While the community began their activism in data ethics and countering algorithmic bias, design justice seeks to retool for many dimensions of justice, including environmental and climate justice \todo{[153]}. As a strategy for advancing social justice, retooling emphasizes the collective effort of sociotechnical development that does not rely on one amazing savior figure, whether they are an extraordinary person or a fix-all tool. Rather, by many people collaboratively developing a toolkit, each piece of a retool represents and propagates the toolkit's values (e.g. antiracism) through a larger system. Continuing Winner's inquiry into design politics, scholars such as Ruha Benjamin analyze the role of technology in systemic problems like racism---which is, that technology is the system \cite{benjamin_race_2019} --- and the lens of retooling offers \revision{appropriately systemic approaches} for solutions. Much of the work in designing for sustainability, representing HCI and many other fields, could be framed as retooling for the cause (sustainability) by promoting aligned values, practices or other supporting causes, such as reuse and recycling across global communities in ereuse.org \cite{franquesa_circular_2016}, urban foraging \cite{disalvo_fruit_2017}, ``circular" design methods \todo{[141]}, individual engagement with sustainability \todo{[39,41]}, deep awareness of the health of one's local ecosystem \todo{[8,73,78,149]}, and critical reflection and deconstruction of existing consumerist processes \todo{[19,101,107,117]}.

Coproduction has been taken up in several disciplines to describe collaborative dynamics between agents, largely between human agents. The earliest applied usages can be found in policy and governance in the ``coproduction" of policy between government and non-government (citizen) actors \todo{[32,103]}. In other social spheres, organizations and communities have coproduced work from healthcare services \todo{[121]} to revitalizing Indigenous weaving cultures \todo{[10]}. These coproduction methods are often cited with participatory design and co-design, acknowledging that the resulting social structures represent how humans do not make things without also making meaning \todo{[66]}. Coproduction can also refer to interactions involving nonhuman agents. Process and operations engineering, refers to coproductions between different nonhuman agents: a ``coproduct" is the integrated product of two or more different processes or supply chains \todo{[3,83]}. Yet as technology begets nonhuman agents who can increasingly reason and communicate like humans (e.g. AI, robots), ``coproduction" is also increasingly applied to working dynamics between humans and nonhumans. Mixed human-nonhuman coproductions are described in collaborative robots, human-robot interaction, Industry 4.0, and mixed manufacturing systems \todo{[30,42]}. We see that ``coproduction" gives us vocabulary to acknowledge and address multiple stakeholders in an emergent practice, including both humans and nonhumans (more-than-humans). STS scholarship suggests that discussing technological and social development in coproductive terms can yield ``better, more complete descriptions of natural and social phenomena" that ``provide normative guidance" or facilitate ``critical interpretation of the diverse ways" which society, technology, and nature influence one another \todo{[66,67]}, which may possibly lead to predictions of more desirable configurations to coproduce values such as equity and sustainability.

% As mentioned in this section's introduction, e-textiles offer a rich space for designers from those working on next-generation wearable devices [48,133] to others creating speculative smart garments for the future of fashion [93,124,127]. My intent for this section is not to prioritize these applications in some order or justify their significance to social progress. Rather, I want to spotlight how e-textiles, as a design domain, is itself a space of social development that contributes to broader discussions of technology's role in society and society's influence on technology. In some form or another, artists, craftspeople, and designers have been exploring combinations of textiles and computing interfaces at least since 1998 [106, 115], and likely earlier. \todo{cite Posch Victorian smart curtain} Looking even further back in history, we find people sewing, weaving, and knitting with metallic fibers in pre-industrial times [40,80,147], and even further back into ancient and prehistoric times, we find people patterning their textiles with symbols and sequences of knots [11]. These methods signified social status, recorded events, or itemized objects---encoding and transmitting data, both qualitative and quantitative, into fabric, just as we do today in silicon.

% Continuing this lineage, a notable thread in e-textiles design explores the expressive, intimate, and sentimental possibilities of the technology. Combining how open-source hardware has made physical computing accessible to a general audience with beginner-friendly craft techniques developed in the e-textiles community, Buechley and Eisenberg especially designed the Lilypad Arduino for prototyping electronic circuits on fabric and nurturing creative, diverse practices \cite{buechley_lilypad_2008}. The wearable microcontroller has large holes to replace conventional header pins, which enable someone to easily hand-stitch a circuit with needle and conductive thread. This model, also adopted by other prototyping boards from Adafruit and BBC's micro:bit, has made these boards effective for classrooms to teach electronics [69,72,116]. Furthermore, since so many human cultures have long histories of textile crafts, e-textiles learning activities are a promising component of culturally responsive teaching methods \cite{kafai_ethnocomputing_2014}, as Kafai et al. have explored in computing education research.
% Outside of formal learning institutions, the expansively-defined e-textiles community has homes in several spaces, especially in DIY art practices. The E-Textiles Summer Camp [151], which then spun off the E-Textiles Spring Break, has created an international creative community focusing on DIY and artistic possibilities for the medium, which then also contributes to the literature. These works include Hannah Perner-WIlson's ``kit of no parts" \cite{perner-wilson_handcrafting_2011} and Irene Posch's explorations of e-textiles tools [113,114], honoring the tools and techniques in e-textiles that are often the simplest, humblest, and most overlooked while questioning assumptions about which technologies and problems are actually ``new", and thereby worthy of attention. This focus on marginalized sites of knowledge and technological prowess is especially evident in Rosner's retelling and community-based co-exploration of the production of woven core memory [123]. As we see from this lineage of invisibilized voices in e-textiles, we uncover a much richer narrative beyond ``novel" futures that is stitched through art, education, and building collaborative communities.


% \subsubsection{Concerns for Sustainability}

% As researchers in human-computer interaction (HCI), fashion, and other disciplines have noted, a future e-textiles industry could potentially contribute to both electronic waste ("e-waste") and textile waste, two of the largest global industrial waste streams [50,98,125]. For today's global textiles industry, the National Resources Defense Council describes textile mills as producing 20\% of the world’s industrial water pollution (through processes of dyeing, washing, etc.) [102] and the Ellen MacArthur Foundation reports that \$500bn is lost each year on ``underused clothes and the lack of recycling". As for the other industrial waste stream, the global electronics industry generates nearly 50 million metric tons of electronics waste or “e-waste" annually [50]. As a category of waste, the problem of e-waste has created secondary problems of regulating, transporting, and properly disposing of it, exacerbating inequities between developed and developing countries as the latter disproportionately receives e-waste to process [29,119]. Even as an ``emergent" industry without many products currently on global markets, e-textiles are still complicit in these problems – the field is already participating in existing industry structures via its development efforts. Researchers have established that two key processes in technological development, prototyping and manufacturing, are significant contributors to material waste [17,36,143]. The field of e-textiles inherits the legacy of manufacturing paradigms (which textiles have historically advanced) and industrially-driven climate change, and thus has a particular responsibility to attend to sustainable values, design, and development.

%\section{Weaving}

%\section{Knitting}

%\section{Yarn Spinning}

%\section{The Textiles Stack}

%\section{Comps: Literature Review}
% The research I describe in my dissertation falls across the intersections of many domains, including HCI, design and artistic practices, STS, and history. To situate our inquiry into the intersection of these intersections, we will review key areas of research in e-textiles, taking an expansive definition that includes related terms such as ``e-textiles" and considers pre-modern textiles as already ``smart". I will motivate my stance on ``retooling" and ``coproduction" by reviewing scholarship that establishes the language of (re)tooling and hybridity in designing technology.


% As stated earlier, e-textiles synthesizes electronics and textiles technologies, both of which are broad fields that encompass many disciplines, paradigms in design and development (e.g. mechanization, hardware/software tools), features in modern infrastructure, and culturally-embedded social practices. Work in e-textiles design can pursue many avenues and take many forms. Consequently, designers discuss work using a variety of terms, which influence their designs and ultimately, the e-textiles technology which result.

% In the development of emergent technologies such as e-textiles, the descriptors ``hybrid" and ``novel" are often used for the designs, components, and systems produced. For instance, one term that largely overlaps with ``smart textiles" and ``e-textiles" is ``flexible hybrid electronics". [126] to emphasize the novelty of integrating digital electronics with flexible substrates (e.g. textiles) for contemporary consumer devices. In HCI design research, ``hybrid craft" is a term often used to describe projects involving craft and computation (e.g. [26,88,138]), many leveraging parametric design (e.g. [44,64,88]), as well as representing ``retooling" as discussed in the previous section. As Devendorf and Rosner explore in ``Beyond Hybrids" [34], these ``hybrid" discourses tend to treat their practices as ``new" and distinct from their precedents, while other design metaphors such as ``intra-action" or ``coproduction" can surface other dimensions for a design practice, such as cultural contexts. They first presented craft coproduction as an alternative to ``hybrid" metaphors, building off Haraway's use of coproductions to describe the mutual shaping of categories and boundaries (e.g. software/hardware, human/machine), they argue for technologies that can explore, rather than resolve, intersections between things we tend to see as ``different." These generative metaphors surface certain design possibilities and bring perspectives from other scholarship to a practice. Namely, coproduction in HCI can offer a greater sense of different agencies, both human and non-human, and how they bring continuing legacies into the process of designing technology than ``hybrid" discourses which can focus on a novelty that disrupts past designs, end-user goals, and features in a technology.

Returning to the specific case of e-textiles, Devendorf and Rosner give examples of how ``coproduction" animates different design concepts and possibilities from ``hybrid". For one, acknowledging how craft and computation are historically intertwined in coproduction yields an exploration of how hand-weaving core memory modules for the Apollo space program were an early e-textiles practice. In fact, a present- and future-centered metaphor such as ``hybrid" would also lose many other past instances of how textiles were crucial to the development of digital technology, as the authors of The Fabric of Interface \todo{[95]} explore --- textiles may in fact form the basis of design metaphors in our modern digital interfaces. Furthermore, contextualizing e-textiles as a coproduction creates space to consider the rich history and nonlinear reality of technological progress. We may find that the contemporary industrial focus on cotton textiles, globally, resulted from shifting away from a previous focus on linen textiles driven by European colonizers' desire for Indian cotton \todo{[13,135]}. HCI practitioners today may find lessons from ``past" technology such as wooden Jacquard looms \todo{[47]}. These examples highlight how we humans have collectively made trade-offs in our technology that were not necessarily ``optimal", which situates inquiries into technological futures.

Much of the retooling in sustainable HCI to foreground environmental impact might also be seen as foregrounding human-nature coproduction. Under a broader theoretical framework of the Anthropocene, researchers have explored sustainable behavior by designing ways to bring users in greater physical contact with the environment \todo{[77,82,85]} or approaches that question the fundamental orientation of HCI as on that is focused on ``ease of use" \todo{[82]}, making space for people to reflect and act with their environment as a coproducer, however challenging and uncomfortable it may be \todo{[35,81,86]}. Specific to textiles and fashion, designers are generally aware of the unsustainability of the global industry and engaging with the aforementioned sustainable design strategies \todo{[23,75,140]}. e-textiles practitioners have explored the unique affordances of textiles for sustainable design tactics, such as the ability to repair textiles-based technologies by darning \todo{[68]} and inherent structural compatibility with designing for disassembly (\revision{explained further in Ch. \ref{ch_unfabricate}}). Designers in e-textiles (and more broadly in textiles/fashion) recognize the particular social dimensions in which their artifacts are situated, so research on sustainable textiles often emphasizes personal relationships, intimate bodily contact, and sentimental value -- for example, Fletcher's Craft of Use \todo{[49]} and Kuusk's work on service-based, personalized production of sustainable e-textiles \todo{[76]}. Expanding beyond individual practices, sustainable textiles has found homes in distributed efforts coproduced between artists and industry entities, including the EU Wear Sustain Network \todo{[55]}, as well as textile waste marketplaces like Queen of Raw \todo{[156]} and Fab Scrap \todo{[157]}.

Coproduction, in questioning how we make meaning while making our things, gives us language to ponder: what does it mean for textiles and their machines to be ``smart"? Ascribing intelligence to things implies a hierarchy of information and experiences which becomes encoded in our technology \todo{[56,65]}. These encodings consist of race, gender, class, and other oppressive structures, manifesting in algorithmic biases that replicate history's injustices. In working with textiles in particular, these patterns include the marginalization of ``craft" as feminine or queer work without technical merit \todo{[11,22,54,130]}, and constructing ``traditional" embodied knowledges as backwards and low-value, as in the case of exploiting Din\'{e} (Navajo) women's labor and weaving expertise for semiconductor manufacturing \cite{nakamura_indigenous_2014}. Through the lens of coproduction, I would argue that textiles have been smart all along, if we truly acknowledge the work that has been put in through millenia to develop this technology.

We can see that coproduction gives us language to animate and guide processes of retooling and designing e-textiles that ``hybrid" descriptions and other innovation discourses often leave unexplored. Combined with a notion of retooling that considers the specific influences of tools on their social-political-technical ecosystems, coproduction conveys the non-deterministic, dynamic nature of the tools' contexts. For myself as a design researcher, I find an environment cacophonously alive with the voices of materials, crafts throughout time, and any involved plants/animals/machines/others. 

\input ch_2_adacad.tex

With this multi-disciplinary conceptualization of ``retooling" in e-textiles, I see opportunities for both ``retooling" as technical invention and ``retooling" as advancing social justice, doing both without choosing between one or the other and embedding values of equity and sustainability in the technology as it is still nascent. In this ecosystem of continual retooling, iterative collective hacking, and negotiating the values embedded therein, what are these values up for negotiation then? In the next section, I will review frameworks for assigning meaning and value under which designers operate, especially those combining technological disciplines.
