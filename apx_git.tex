% glossary
\chapter{Git Repositories}
\label{apx_git}

\singlespacing
\textbf{Ch. \ref{ch_adacad}} -- AdaCAD
\begin{itemize}
  \item Source code for the latest version of AdaCAD. \url{https://github.com/UnstableDesign/AdaCAD}
  \item Detailed draft and notes for the Multi-Component fabric. \url{https://adacad.unstable.design/multicomponent/}
\end{itemize}

\textbf{Ch. \ref{ch_e-textiles}} -- Naming E-textiles
\begin{itemize}
  \item Python notebook (Jupyter/Colab) of the survey response analysis. \url{https://github.com/sminliwu/ETextiles-Language}
\end{itemize}

\textbf{Ch. \ref{ch_loom-pedals}} -- Loom Pedals
\begin{itemize}
  \item Node.JS source code for driver software (Raspberry Pi or a personal computer) that controls the Pedal modules, AdaCAD, and a Jacquard loom. \url{https://github.com/UnstableDesign/Loom-Pedals-Driver}
  \item Design files for the physical components: CAD files for physical enclosures; schematics, testing circuitry, and PCB designs for Pedal modules. \url(https://github.com/UnstableDesign/Loom-Pedals-Hardware)
  \item Angular module for manually installing the Draft Player interface for the Loom Pedals into a local copy of AdaCAD. \url{https://github.com/UnstableDesign/Loom-Pedals-AdaCAD}
  \item A branch of the main AdaCAD repository which runs a version with the Draft Player. \url{https://github.com/sminliwu/AdaCAD-weaver/tree/pedals-chi}
\end{itemize}

\textbf{Etc.}
\begin{itemize}
  \item The source files for this very dissertation: LaTeX, BibTeX, modified class style from the CU Boulder template, modified bibliography style from ACM. \url{https://github.com/sminliwu/Dissertation}
\end{itemize}

% The definitions in this appendix are not intended to be authoritative or universally applicable. Textiles span multiple cultures, languages, communities, historical eras, and technological paradigms; the English language only covers a sliver of these contexts.

% \section{Weaving}

% \begin{itemize}
% \item \defineterm{weaving}{crossing two or more sets of yarns in different directions in order to form fabric.}
% \item \defineterm{warp}{\textit{(noun)} one set of yarns used in weaving, conventionally represented as vertical. The warp yarns }

% \end{itemize}