\chapter{Conclusion}

% I think concluding sections in academic writing should be useful. So if you've made it through the rest of this dissertation up to this point---first of all, thank you for your time. Secondly, I don't know what specific use you'll get out of my writing, so I will adopt the study guide approach and write as if you will be quizzed someday about the contents.

\revision{To conclude, I am reflecting upon this entire dissertation as: 1) the largest document I have ever put together in my life; and 2) a milestone capping off this phase of my life while also pointing to future roads ahead.}%\todo{finish writing paragraph AFTER rest of conclusion}

\section{Summary}

% to review the themes:
% CRAFT first
This dissertation maps out a design orientation that is not a specific tool, but a set of guidelines that if a tool follows, will implement a craft-based approach to e-textiles sustainability. 
% \revision{The research projects shown in this dissertation present 
% On a conceptual level, }
I define sustainability as the ability for a community to live in collaboration with their environment, including non-human animals, machines, and the local ecosystem. While these are all very heady concepts, designers already work with the products of \revision{an unsustainable world}: manufactured yarns, wires, existing equipment, etc. Thus, designers can directly engage with all of these factors not through thinking and theorizing about them, but through a \keyterm{craft} mindset that emphasizes how their hands are materially interacting with things.
% what do I want the reader to take away

% craft-based approach to e-textiles design tools, which I have developed through my PhD research. I specifically break down this design orientation as ``retooling for coproduction'', combining two key concepts that shaped my thinking on how craft engages with the technical and the social. 
This design orientation is called ``retooling for coproduction" in reference to two key concepts. The first, \keyterm{retooling}, is a strategy from the design justice community for dismantling systemic oppression within technology by looking at the whole toolchain (datasets, manufacturing, user testing, materials sourcing, etc.) for biases \cite{costanza-chock_design_2020}. The second, \keyterm{coproduction}, is a concept from feminist design and technoscience studies that metaphorically describes how entities (e.g. technology and society, designer and user, craftsperson and tool) mutually shape one another, and that this shaping is also subject to other influences \cite{jasanoff_states_2010,haraway_staying_2016}. 

\revision{Through Chapters \ref{ch_e-textiles}--\ref{ch_speculations}, }I present \revision{four} studies investigating e-textiles tooling as examples where this orientation generated novel connections and design opportunities. To recap each of their research questions:

\begin{quote}
\revision{\texttt{Ch. \ref{ch_e-textiles}: Naming E-Textiles}}

\revision{How do designers shape e-textiles technology through their language choices?}

\vspace{1em}
\texttt{Ch. \ref{ch_unfabricate}: Unfabricate}

How can e-textiles be designed for disassembly and recycling?

\vspace{1em}
\texttt{Ch. \ref{ch_loom-pedals}: Loom Pedals}

How can Jacquard weaving be retooled to promote coproductive improvisation?

\vspace{1em}
\texttt{Ch: \ref{ch_speculations}: Speculations}

How does sustainability show up (or not) within the values of e-textiles practice? How could e-textiles materially engage in \textit{doing} sustainability?

\end{quote}

% The research undertaken during my PhD studies, which focuses on woven e-textiles, demonstrates some components for retooling e-textiles for a more sustainable future by engaging craft. 
\revision{These projects are retoolings of e-textiles language usage, structural designs, fabrication machinery, and speculative tactics respectively.}

% Elements of craft are present in many disciplines, manifesting in human interactions with materials, information, and the more-than-human world. Likewise, wicked problems such as sustainability leak into all aspects of the human world, from our clothing Thus, by picking up these craft threads, braiding and retooling them, a designer can connect the technical aspects of their work to broader

\section{Contribution and Limitations}

\revision{Taken together, my inquiries give examples of how craft offers ways to design and manufacture differently. In our present time of climate change and global ecological collapse, we certainly see an urgent need to do things differently, i.e. more \textit{sustainably}. Craft speaks to the deeply human activity of making tools, materials, and structures with our hands and with each other. In this framing, our contemporary model of ``manufacturing'' is only one such possible path for making things --- a reassuring statement of \textit{yes, there are alternatives} when the aforementioned problems of unsustainability feel impossible to solve.}

\revision{E-textiles presents specific challenges, but also opportunities, for the design and manufacturing of sustainable future technologies. In each of my projects, my knowledge and experience as a craftsperson (rather than an engineer or scientist) guided the research to these opportunities. In Ch. \ref{ch_e-textiles}, ``Naming E-textiles'', the language survey results highlighted the plurality of design perspectives in e-textiles. While we could call for a sort of consensus or standardization in response to the survey, we could also take an alternative approach like many textile crafts: accept convergent thinking from multiple communities and recognize that there are many names for the same thing (e.g. looms and needles). Continuing to ``Unfabricate'', my entire exploration of unravelling textiles would not have been possible without the existing work by DIY communities in recycling yarns. In Ch. \ref{ch_loom-pedals}, ``Loom Pedals'', I explicitly chose traditional shaft looms over modern engineering products such as automatic Jacquard looms as the model for my system design. Finally, in ``Speculations'', we reflected on the landscape of e-textiles prototyping and manufacturing in order to generate design tools for practicing sustainability. Our speculations included social spaces and organizing tactics as tools, beyond conventional notions of hardware, materials, and software tools. By looking to crafting communities, such as quilting bees and knitting circles, we realized that the things we make are not only shaped by the tools and materials in our hands, but also the people and spaces around us.}
% what I think is the most valuable part of the work presented, and which parts are weaker/could grow

\revision{My contribution to e-textiles research is this synthesis of craft, retooling, and coproduction to serve sustainable design practices in e-textiles. One part consists of the ``what'' --- the tools I developed during my PhD for anyone to use.}

% In the studies I discussed in Chapters \ref{ch_e-textiles}-\ref{ch_speculations}, these tools are generally the primary outcome. However, I do note that some of these tools were side effects of hacking other 

\begin{itemize}
  \item \textbf{\revision{Naming:}} A domain-specific language survey as a cultural probe 
  \item \textbf{Unfabricate:} A workflow for designing, fabricating, and disassembling woven e-textiles 
  \item \textbf{Loom Pedals:} An integrated hardware/software interface for improvisational, experimental weaving
  \item \textbf{Speculations:} A tactic of ``speculative construction'' to generate concepts for tools to serve a social justice agenda, e.g. sustainability
\end{itemize}

\revision{Another part of my contribution consists of the ``how'' --- the design methods and tactics employed throughout my learning process in order to arrive at the ``what''. My main strategy could be framed as \textbf{translating} knowledge between communities, which poetically fits the repeating theme of language in my research, and my upbringing in a multilingual, immigrant home community. Through my educational background, I have one foot in DIY textile/fiber crafts and another foot in engineering and lab science. I blend these two discourses in order to translate textiles craft knowledge and values to new audiences, notably HCI researchers in design and fabrication. The resulting insights show the value of considering craft in a sustainable design practice, suggesting an approach to sustainability distinct from, but complementary to, other approaches that focus on policy-making, material development, lifecycle analysis, and many other factors.}

\revision{At this point, my dissertation only demonstrates that a craft-based design approach of ``retooling for coproduction'' towards sustainable e-textiles is \textit{productive} in the case of one design researcher: me. Productivity, in this case, would be defined as the ability to generate new design possibilities. However, the research has not yet seen whether my design orientation is productive for other designers. How would someone with a different educational path than I translate between craft, textiles, and design? Furthermore, is this approach \textit{effective} at translating craft into a sustainable design method, in terms of transmitting values of retooling, community, and hands-on work? We will have to revisit these questions, and many more, in future research.}

% Taken together, these tools certainly all form a collection; but I hesitate to call them a ``kit'', as that would imply they could be used together for a targeted purpose. Instead, I have assembled parts of these tools into an overarching method of researching technology through \textit{craft}. 

\section{Future Directions}

My PhD research has largely taken place in a university lab setting as a result of this tools-focused orientation, and I aim to explore the social dimensions of technological development in my post-dissertation work.
In the words of a friend, I am an ``idea factory''. So to keep this section for research ideas relatively contained, \revision{I will make sure to ground each proposal in projects which I have already built or carried out.}

\subsection{Crafting Sustainable E-Textiles Infrastructures}
% biofoam represents start of this?

I am not content leaving the speculative elements of Chapter \ref{ch_speculations} as speculative. In the spirit of a more activist-oriented speculative approach, I explicitly target the preferable futures with my design practice and will do everything in my power to steer towards them. Thus, I would actually be interested in attempting to implement e-textile interposer components, organizing some sort of e-textiles sustainability working group, and facilitating an inclusive community space. In other work outside of this dissertation, I have worked with colleagues who deal more with material development and biodesign, thus dipping my toes in the development of sustainable materials for e-textiles. 
By retooling in a more systematic manner to encompass an integrated sustainable e-textiles toolchain, I hope to engage with the unfamiliar topic of infrastructure.

\subsection{Retooling in a Community}
% people

As mentioned above and with the Loom Pedals, my research practice has been unfortunately disconnected from a steady community. Crafting, retooling, and coproducing are all verbs which take place amongst other agencies in design. At this point in my career, I feel a distinct need to engage with more social research methods and collaborate with people to develop my own perspective. Related to the previous direction's infrastructure focus, I am drawn towards Richard Wong's method of ``infrastructural speculations''. 

This method would be particularly interesting to carry out in a group setting, creating of a speculative artifact that represents sustainable futures in e-textiles by surfacing and honoring the ongoing practices in \revision{crafting communities (i.e. \textit{Weaving Hacks})} that are presently marginalized. The artifact will be produced collaboratively, and in fact will consist of multiple linked components that form an infrastructural speculation, borrowing from the framework of \cite{wong_infrastructural_2020}. The artifact's foundation will be the collaboration process itself -- the shared space formed by myself and my collaborators through our correspondences and evolving relationships. The specifics of what this research-creation will produce in terms of publication, documentation of novel techniques, and tangible tools and products.

% Figure 2. Representation of discrete artifacts from each collaborator with possible lines of dialogue or connection that form throughout the project. This diagram contextualizes each collaborators' contribution to the project as arising from their own distinct ``worlds of practice", defined by Takhteyev as ``systems of activities comprised of people, ideas, and material objects, linked simultaneously by shared meanings and joint projects." [55]

% A Garment and Their Parent Loom
% Modularity is an established design tactic for sustainable products [36,59,60], taking the form of garments and devices with interchangeable components. Some of this modular design extends beyond fabricated products to collections of such products (e.g. capsule wardrobes [9,17] and the machines that produce them [13,29,35] The artifact which I will propose to my collaborators will be a two-part system with modularity as a core design principle: an explicitly modular loom and a modular smart textile garment woven by the same loom.
% In order to reduce the extractive impact of electronic hardware manufacturing, I posit that  computing machines especially need to be designed for modularity and reusability in order to be sustainable. What's the relationship between modularity for machines and sustainability in computing? In semiconductor device manufacturing, the most energy- and carbon-intensive stage is extracting raw resources [cite STMicroelectronics]. If we also consider the locations of these extractive operations, with minerals from all six inhabited continents needed to produce a device [31], transportation emissions are also high. Modularity allows us to more effective reuse the resources that have already been extracted.
% The specific smart textile garment that I will produce is a coat or jacket for an outdoor work environment (e.g. farming, hunting/tracking, building). The ``smart" electronic functions will be heating and temperature regulation for the wearer, as well as data-logging of body and environment. Following the expansive definition of ``smart textiles" outlined in Q1, the textile innovations also comprise smart features. The coat will be constructed in a modular fashion for interchangeability, cleaning, and repair as it is used over many years. Each fabric module of the garment will be designed within a system that details the woven structures needed to create the electronically-enabled fabrics, as well as fabrics optimized for insulation, moisture-wicking meshes, and abrasion resistance. The modules can be stacked in a layered fabric assembly like quilts, where top and bottom ``face" fabrics sandwich an insulating ``batting" layer. Finally, these modules will be assembled with a linker, a technique used in fully-fashioned knitwear garments to create seams that ``unzip" to disassemble. More information on disassemble-able structures in knitting and weaving can be found in my previous work on Unfabricate. [65]
% I have chosen to weave a coat as an exemplar object of future sustainable smart textiles as it relates to several dimensions of sustainability referenced in Q1. As an outer layer garment, a coat is an interface between a human body and their environment. The electronic functions of the smart textile coat are designed to work alongside this existing relationship between a body and outdoor environment, rather than supplant or separate the body. Furthermore, the aesthetic references for this coat's design include vintage workwear, sashiko, and patchwork quilts, referencing histories of continual use, repair, modification, and reuse which carry garments across generations and into the present.
% The other half of my speculative infrastructural component, the loom built to create the garment, will be modular so that all of the modules of the coat can be created on the same machine. The idea of designing a machine specifically for its product is inspired by the idea that our designed creations are our descendents much like biological children, and that an anti-colonial mindset needs to design for future generations of kin [20,26]. This loom will be modestly sized for tabletop operation, and like a pin loom or tapestry loom, the core of the loom is a simple frame that tensions yarn (the ``warp") over a small working area. Using interchangeable modules which assist in manipulating the warp in different patterns in order to weave other yarns (the ``weft"), the weaver will be able to use the loom to produce the different types of fabric modules needed for the coat. The first module will be used to produce a structure which I created in my previous project, Unfabricate, which allows each fabric module to be unravelled and recycled for yarn after use. Another module will assist in twisting the warp to achieve a mesh-like lace structure for breathable fabrics. The final module that can assist in creating the coat will use a linking mechanism to attach an already-finished fabric module to the fabric on the loom. Other modules may be formulated and created as needed, for example, to assist with mending fabric circuits or disassembling modules.
% The control system will consist a set of programmable pedals or buttons which are configured for the weaver to quickly input and execute their most favored woven structures. This embedded interface is an extension of another project which I contributed to, AdaCAD, which focused on designing software interfaces for woven smart textile design. Implementing similar principles from the software tool, the hardware interface will blend the design phase of textile creation with the fabrication phase. Users will be able to transition from editing their design file on a screen to ``executing" the file on the loom, where the interface and loom hardware assist in raising/twisting the warp and assist the weaver with prompts for hand manipulations. Continuing in the vein of adjustability and customizability, the weaver will be able to control the degree to which they are executing the file, versus the loom. As Bauhaus weaver and design scholar Anni Albers observed, woven structures can be accomplished through multiple means, and even complex structures can be done on simple looms with hand techniques. Some weavers may value the hand manipulation and mentally tracking their progress through the design, so the loom will be completely usable without electronics.
% The negotiability embedded within the loom-garment system goes back to my emphasis on collaboration as a vital aspect of sustainable e-textiles practice, including collaboration with one's environment, workshop, and tools. 
In acknowledging one's local ecosystem of material agencies and practices, we can begin to build an infrastructure through bricolage as the bricklayer does \cite{turkle_epistemological_1990, vallgarda_interaction_2015}. I see built objects as anchors for spaces and provocations for further dialogue, which facilitate group conversations that acknowledge a multiplicity of viewpoints. Taking a brick from DiSalvo's adversarial design framework \cite{disalvo_adversarial_2012}, these negotiations with other perspectives is my way to create systems and spaces for sustainable practices and communities.

\subsection{Handmade Ecologies}

While being a PhD student has been the longest position I have held professionally so far, it is still nearly not enough time to design something and \textit{live with it}. Several side projects of mine involve caring for other living beings. From carrying a plant as an environmental sensor, to hacking an automated door for my backyard flock, I believe that digital technologies can find an ecological niche just as humans should find one within their local context. Perhaps this research would not follow normal grant and publication timelines. Perhaps it would obey the seasons, the weather, and the hours of daylight. 

\begin{quote}
  \noindent\textbf{Winter:} a time of burrowing into a shelter and reflecting upon one's reserves, planning for a time that is more conducive to growth.

  \noindent\textbf{Spring:} a time for sprouts to develop rapidly and cast their growths across wide areas. 

  \noindent\textbf{Summer:} a time for those who took root well to develop their systems to the fullest potentials and keep their deepest roots hydrated.

  \noindent\textbf{Autumn:} a time of harvesting and feasting, a time for sharing the bounties of labor with a community.
\end{quote}

When winter comes again, the reserves from the harvest come to rest inside, for us to ruminate upon and take stock of lessons for the next growing cycle. What would prototyping look like in such an explicitly cyclical nature of inquiry that periodically returns to deep reflection? 
% In another keystone discussion as in the beginning of the collaboration, we will conduct another group brainstorm to revisit all the produced documents throughout the prior phases of the study. Reflecting on the unventions and machine modifications that we made in the cycle, I will focus the discussion on how these actions opened up further opportunities for inquiry. 
These discursive tactics draw from both reflective design practices, as well as critical fabulations as formulated by Rosner et al. \cite{sengers_reflective_2005,rosner_critical_2018} for tactics to probe pluralistic futures and histories by interrogating the present. In adapting these machines and influencing human practices, how can this cyclical engagement promote continual iteration that could scale from discrete relationships to a future infrastructure? The questioning here seeks ways of iteratively progressing from an unsustainable way of being and making.

% (Q4) Collaborators
% Having introduced my personal orientation towards sustainable e-textiles, and acknowledging the limits of an individual framework in working towards a pluralistic, sustainable world, this project must involve other humans. My collaborators and my relationships with them must be foregrounded before any of our future work. I have identified three distinct relationships that embody aspects of collaborative design and prototyping which my collaborators represent. Furthermore, the artifact which I proposed in the previous section has been designed to appeal to these collaborators' own research interests or be potentially symbiotic with their ongoing work.
% Any resemblance to real persons is completely intentional and not coincidental.
% User Zero, or the Co-Futurist
% My own design research ethos centers designing for marginalized perspectives, and my previous work with autobiographical design methods [33,65] has used my own perspective and identities as material for speculating on sustainable futures. However, to explore plural narratives of marginalization and achieving justice, I need my first collaborator, User Zero, to share some aspects of these experiences and aspirations as a Black/Indigenous/Person of Color (BIPOC). User Zero already has an established personal relationship with me through past collaborations on artist-activist work, research in its own right though not necessarily legitimized as such. I would also consider them part of my chosen family, a kinship built on mutual accomplice-hood and sharing living spaces.  
% User Zero, or my Co-Futurist, works outdoors in the majority of their days as a community organizer in their Indigenous community, food justice activist, and agricultural worker. While they do not have specific experience or formal education in electronics or e-textiles, they have professional experience in factory workshops, practice textile handicrafts such as embroidery and tapestry weaving as hobbies, and have always engaged in hands-on building. Each of us has expressed at multiple points learning specific technical skills from each other, such as cultivating mushrooms and soldering. Building my artifact in conjunction with theirs gives us an opportunity to achieve this long-standing goal. In fact, if they express interest in working with me to co-design the modular loom and garment (and share the garment with me in our lives), then I would be more than happy. Their participation as an equal collaborator in academic research would help gain the recognition they deserve for their expertise.
% The Tool Maker
% In contrast to User Zero, the Tool Maker makes their living in designing and creating devices as products that enrich their customers' creative and technical practices. While they may or may not explicitly identify with the Maker movement because they perceive being a capital ``M" Maker as technocratic-3D-printing-lasers-centric while textiles have a ``low-tech" connotation [28], they are passionate about emergent making and craft-based discourses. The Tool Maker finds themself at the fringes of interdisciplinary technological discourse, as many often do in such hybrid discourses like e-textiles, bioart [4,40,56], and electronic music [9,19,20]. 
% The Tool Maker works as a designer in a loom and spinning wheel manufacturer, Company X. While I personally have not directly collaborated with anyone in Company X, there are existing relationships between my research group and the company, as well as an active engagement with the handweaving community local to both of our organizations. The current collaboration involves developing open-source weaving computer-assisted design (CAD) software, evolutions of AdaCAD [12], with their guidance and embodied knowledge of weaving tools as both a user and designer. This software project is in exploratory discussion to tie into a future hardware integration as a physical module for the company's loom products. Combining my own artifact's loom hacking would take advantage of our works' synchronicity and probe a wider diversity of e-textiles weaving practices.
% The Enabler
% This collaborator would not necessarily be involved in the hands-on making of any artifacts, but would importantly bring vital contextual knowledge as a consultant on sociological knowledge such as supply chain analysis, museum or gallery curation, and managerial techniques. In a study on challenges to scaling electronic hardware production, Khuruna \& Hodges identified the ``enabler" as a distinct role in bringing a prototype to contexts outside of a lab or makerspace, supporting the ``creators'' or makers involved physical design, circuit implementation, and other details. [22] While topics such as supply chain management and economics are available to study in academic institutions, such ``non-technical" knowledge is often gained through first-hand professional experience, even more so when transitioning to different disciplines as will occur in e-textiles development. As such, the Enabler is in an advanced stage of their career, having seen through and been involved with several years of technological transitions. 
% There are two possible candidates for the Enabler. Enabler A is a veteran textile designer who has transitioned from working on individual design projects in their company to managing projects that represent the innovation and experimental development arm. At the conclusion of my previous study in which they were a participant, they expressed interest in exploring the idea of future development spaces in e-textiles. Enabler B is a weaving instructor, practicing artist, and part-time faculty in a university textiles program with connections to weaving manufacturers focusing on supporting artistic practices. B has much experience organizing and creating residencies for interdisciplinary artists, continually bringing new perspectives into their community of practice.
% (Q3) Proposed Study
% Timeline Overview
% The seasonal titles for the proposed phases reflect the permeable boundaries between different segments of this research. My hope is that they also suggest a cyclical future and nonlinear timeline for future work that could spring from the study. 
% The timeline includes how I will have dialogue with my collaborators, as well as the build phases of how I create my individual artifact. Each phase has deliverables outlined as goals: one is almost entirely individual work for myself, the other is produced as a collective.
% Phase/Season
% Research Question
% Primary Methods
% Winter (2-3 months)
% How do different stakeholders in sustainable smart textile development bring their experiences and specific technical knowledges to their futuring?
% Bodystorming

% Spring / Summer 
% (6-7 months)
% What are commonalities and differences in prototyping approaches across electronics and textile practices, and how are historical or contemporary tools and machines transformed through prototyping?
% Unventing
% Critical making
% Fall (3-4 months)
% How do modifications and adaptations to a fabrication process transmit to a maker's communities of practice?
% Autoethnography
% Winter (2-3 months)
% Where do machine modifications open up further opportunities in adapting the machine, influencing human practices, and speculating on emergent sustainable infrastructures? Particularly in sustainable e-textiles, how do such adaptations unvent and bring to surface other instances of modification in related histories?
% Reflective design
% Critical fabulations

% Phase 1: Winter

% Research Question: How do different stakeholders in sustainable smart textile development bring their experiences and specific technical knowledges to their futuring?
% Methods
% This phase focuses on relationship- and collective-building between the collaborators, seeking to create, as much as possible, a non-hierarchical dynamic among the assembled group. The keystone of this phase will be a group conversation/closed workshopping session with all collaborators gathered, virtually or physically in-person. Using bodystorming methods as described in Höök's Somaesthetic Design [18], this meeting will imagine the embodied interactions that would be present in sustainable e-textiles futures that are shared by the group. The conversation will be structured to keep things focused, knowing that there will be room for unstructured, spontaneous interactions outside of the meeting as part of ongoing research-creation relationships. My questions would include: ``What will each person get from this project? What do they see in my proposed prototype that would be useful for them?"
% I will have individual conversations before and after the group meeting with each collaborator to establish one-on-one relationships with me. These conversations will be unstructured, as their primary purpose is to establish channels of communication. Throughout this phase, we will focus on exchanging stories from personal experiences and readings to put our individual frameworks in dialogue. In terms of logistics, we will formulate explicit verbal or written agreements about each others' time commitments to the project, taking into account their other responsibilities.
% Deliverable
% Collective: From the group meeting, we would produce a sketched map that includes my proposed artifact and their relationship to the other artifacts that my collaborators will develop. The artifacts can be components of the loom-garment system that they see worth investing in, but they can also be a related artifact that could be used in conjunction with others in the project. See Figure 2.
% Individual: In my own artifact, I will finalize the design requirements of the final garment (i.e. usability and reliability metrics) and a set of system diagrams for the loom which detail the modules, their sequence of operation, and priority in prototyping. As some modules are iterations upon past prototypes of woven fabrication systems (i.e. Unfabricate, TC2 pedals), this planning documentation will also include the materials from those previous projects. As this phase literally coincides with the winter season, the Co-Futurist and I will both be in our homes more often and will teach each other the necessary skills for our artifacts, such as weaving and outdoor survival in the cold.
% Phase 2: Spring / Summer

% Research Question: What are commonalities and differences in prototyping approaches across electronics and textile practices, and how are historical or contemporary tools and machines transformed through prototyping?
% Methods
% This phase focuses on each collaborator creating an initial prototype based on the identified artifacts or ecosystem components from Phase 1. Using guidelines from critical making methods to engage socio-technical reflections in tandem with hands-on making [40,42], our group discussion questions will center on noticing thought processes as the collaborators make their artifacts, how they draw from other bodies of knowledge, may encounter political issues (e.g. in sourcing materials, coming across a policy limitation), and create workarounds on-the-fly in the tangible process. These questions will aim at developing ``unventing" as a design method. Summer, the latter 3 or 4 months of this phase will shift towards evaluating and iterating upon the work done so far to reach a medium- to high-fidelity prototype. Discussion questions in this second half, while still focused on building, will be more targeted: finding major bugs and points of friction in seams/interfaces, and identifying directions for further growth.
% Deliverable
% Collective: Documentation of machine modifications with instructions to reproduce this initial version, focusing on technique, process, and design heuristics rather than a product. We will identify any commonalities and key differences in our design processes to begin extrapolating potential principles which would guide a group manifesto.
% Individual:  The modular loom's individual modules will be ready to weave swatches throughout the spring, making mechanical adjustments so that they are physically compatible with each other. At the beginning of summer, I will be able to shift from weaving swatches to weaving pieces that are ready for the garment, and assembling the final coat. Documentation of the finished garment and loom will include photographs, detailed instructions for using the loom modules, and an organized source code and fabrication file repository.
% Phase 3: Fall
% Research Question: How do modifications and adaptations to a fabrication process transmit to a maker's communities of practice?
% Methods
% This phase directly corresponds to a traditional user testing phase in a prototyping process, so some collaborators may have existing testing protocols for their own purposes (e.g. Tool Maker may have product testing requirements, Enabler may have organizational precedents for new initiatives) which they will use to test their artifact with their user community. In our conversations, I will also take an autoethnographic orientation towards our procedures, encouraging all the collaborators to also reflect on the user or customer test as engaging with their community of practice. Through questions such as, ``Who do you make things for (who are your users), and has that shifted over time?" and ``Do you see different roles in making, using, etc. among you and your community?" I will lead the group in reflecting on the structure of relationships in our group's context, and how unventions in fabrication processes can shift those relationships.
% Deliverable
% Collective: We will compile an anonymized summary of community feedback with analysis of how the infrastructural speculation artifacts seemed to generate (or not) further engagement with modification and unvention.
% Individual: The Co-Futurist and I will take turns wearing the coat while doing outside work, whether they are out in the backcountry or if we're in a backyard, documenting sensations from the environment, our bodies, and the garment and recording observations of how the coat is functioning and wearing. If we are working alongside other people, we will also record those observers' reactions to the coat and (if they are human friends) initiate conversations about the fabrication and loom construction to probe whether they find the speculative infrastructure meaningful or useful.
% Phase 4: Winter

% Research Question: Where do machine modifications open up further opportunities in adapting the machine, influencing human practices, and speculating on emergent sustainable infrastructures? Particularly in sustainable e-textiles, how do such adaptations unvent and bring to surface other instances of modification in related histories?
% Methods

% Deliverable
% Collective: Acknowledging that there is further work to be done after a very early-stage user test, we will publicly announce the prototypes as an alpha release of collaborators' artifacts. As one more component of the manifesto (see Q5), we will produce a document of possible future directions for research/development.


% time

% PROJECTS: (things that haven't made it into a full paper yet)
% - Ozone Vest
% - Chicken coop electronics
% - Sourdough incubator


% ---> local ecosystem/community

% Theory/methods: design justice, speculative design

% The future I want for my loved ones, friends, family, community --- sustainability is about imagining a world/future, but mine is anchored in people

\pagebreak
\section{Calls to Action}

At this point, dear reader, both you and I have been doing a lot of reading/writing. While writing is itself a craft, I know that it is not anywhere near my primary crafts (says the person who wrote a whole dissertation). If you're like me, you might be itching to get away from a screen, get into your body and hands, and explore futures through physical crafting. I leave you with some challenges:

\begin{itemize}
  \item Hold a listening session with your materials. Where do they come from? Who was involved in their making? What histories are embedded in their physical substance?
  \item Situate yourself in your local ecosystem. Are you on colonized land? Whose ancestors belong to the land you work on? What weeds, pests, and other life-taken-for-granted are common in the area?
  \item Build something for a loved one. What struggles are happening in your community? What tools would a friend/family member need to empower themselves in their life?
\end{itemize}