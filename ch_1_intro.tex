\chapter{Introduction}
\label{ch-intro}

\setlength{\parindent}{0.5in}

% small computers going into everything --> pervasive computing, new devices, more portable
% ``smart'' everything (ubiquitous)
% of these things, I focus on textiles
\par
One trend has held steady for the past few decades: our computers and electronic devices shrink, becoming more portable, more powerful, and more present in our human lives. It is now possible to make computing devices on the scale of nanometers to invisibly embed into our built environment
% which means that theoretically, researchers could integrate sensing, actuation, and connectivity into any part of everyday life
\revision{ --- awakening existing objects as ``smart". 
% Some of these ``smart'' things include: homes, cars, watches, eyeglasses, clothing, and shoes. 
Many of these objects involve some form of textiles, creating a large possibility space for ``electronic textiles", or \keyterm{e-textiles}.}
% We can stitch and weave circuitry onto My research focuses on the things that involve textiles, which can be broadly called \keyterm{e-textiles}, as in ``electronic textiles". 
% More specifically, I am interested in how to design these future technologies with the values of \keyterm{craft} embedded within their roots. 
My dissertation will dig much further into the question of what e-textiles are, but let's start with this working definition:

\begin{quote}
  \texttt{E-TEXTILES}

  A broad category of emergent technologies united by the goal of integrating digital technologies (e.g. sensors, heating/light/actuation, networking) with garments, home objects, and other textile objects.

  \textit{see also:} \texttt{\small{}smart textiles, wearable technology, flexible hybrid electronics}
\end{quote}

\par
\revision{While e-textiles promises many exciting possibilities, these futures are all clouded by the question of whether or not they are \emph{sustainable}. Will these smart garments and seamless soft computers become reality if the current political/economic status quo continues to lead us towards ecological collapse?}
\revision{This climate anxiety has permeated my generation's lifetime, as well as my PhD research. Through this dissertation, I will present my approach to sustainable e-textiles which emerged through \keyterm{craft}. To put it plainly, my research asks:}

\begin{quote}
  \revision{\textit{How could sustainability be enacted through hands-on making?}}
\end{quote}

% how did I get here
\section{Why Craft?}

``Craft'' is a central term in my research. For me, it implies working with my hands, connecting physically to the selected materials, and spending time in this state of making. But I can't define ``craft'' in words. To quote Glenn Adamson in his book, \textit{Thinking Through Craft}, ``Craft only exists in motion. It is a way of doing things, not a classification of objects, institutions, or people.'' \cite{adamson_thinking_2019}. The term ``craft'' is simply just a word, and the idea which the word attempts to name is not something to be defined with more words, but \textit{felt}. 

Looking back, all of my life's work has dealt with craft---though some people may not see the connection. After all, I was tracked into STEM (science, technology, engineering, and math) before I was 10 years old, and went on to study physics and computing for my bachelors degree. It wasn't until I started knitting, a far more conventional ``craft'', that I realized that aligning the lens of an experiment's apparatus and deriving equations on a whiteboard were also crafts. Thus, I came to see e-textiles as a domain where I could practice my machine crafts alongside my textiles crafts, and in fact, come to see them as one and the same.

During my PhD studies, I would further develop my understanding of craft as an essential research method. As you read the rest of this dissertation, you'll see references to methods from reflective design, autobiographical design, design justice, adversarial design, and more. If these seem like fancy, intimidating words, don't worry: I certainly didn't know about them before my PhD, and I still don't fully even grasp what ``design'' is. However, I can quite literally grasp ``craft"---bend it, rip it, bite into it. Guided by other design researchers within human-computer interaction (HCI) and adjacent fields, I connected my visceral, wordless sense of craft with these abstract, academic design terms. Thus, we arrive at this more verbose phrasing of my ``craft": \keyterm{retooling for coproduction}. Let's break that down.

\section{Retooling: Crafting for Sustainability and Social Justice}

Creating and improving tools has driven much of humanity's technological progress -- tools let people make things taller, stronger, more precisely; envision more ambitious designs; even share their ideas across greater distances. That tools make certain tasks possible or easier, while constantly evolving, describes one process by which our technology and societal development are entangled. In engineering disciplines, this iterative process of ``retooling'' mainly describes the literal reconfiguration and/or updating of tools for factories and other mechanized workflows \cite{merriam-webster_retool_2023}. Yet, retooling has been particularly taken up by design justice as a framework for analyzing existing biases in technological design, the social inequities that our designs perpetuate, and how creating new tools or rehabilitating existing ones can ``retool'' sociotechnical systems to advance justice for all \cite{costanza-chock_design_2020}---e.g. retooling social media content algorithms to address the racial, gendered, etc. biases they reinforce to support other forms of activism. Sasha Costanza-Chock, in her book \textit{Design Justice}, elaborates that retooling this particular technology will include developing ``intersectional user stories, testing approaches, training data, benchmarks, standards, validation processes, and impact assessments, among many other tools.'' Specific to my domain of e-textiles, I will define ``retooling'' as:

\noindent\begin{minipage}{\linewidth}
\begin{quote}
  % \interlinepenalty=10000
  \texttt{RETOOLING E-TEXTILES}

  Creating new tools or modifying existing ones from other technologies (including electronics and textiles) to support emergent e-textiles design practices while attending to the fact that these tools will influence values, mindsets, and other tools used by designers, as well as who is centered as a ``designer". Retooling e-textiles may target design software, physical loom hardware, interdisciplinary collaborative practices, embedded interfaces for electronically-enabled looms, and community-generated databases of e-textiles projects, among many other possibilities.
\end{quote}
\vspace{1em}
\end{minipage}

My dissertation argues that, while some of these issues may be addressed with developing new materials or design methods, the e-textiles field needs to also develop appropriate tools. \revision{I choose to use design justice's definition of ``retooling'' to emphasize that this strategy is not ``making'' or ``inventing'' tools to make e-textiles design and fabrication easier or more efficient. Rather, retooling targets the \textit{values} embedded within e-textiles artifacts and applications, so that the designers employing this strategy must interrogate how their work impacts sustainability efforts, however implicitly or unintentionally. The next section will unpack further what these values are when retooling targets sustainability.}
%  for the practice that are rooted deeply within the textiles foundation of the technology. 

% I also see \citedterm{retooling} as a key element of craft. 

% Tools are not only pieces of technology, but they also enable humans to design and develop other technologies. 

% What is e-textiles retooling for? To what end(s) are designers seeking when they integrate electronic functionality (e.g. sensing, actuating, wireless networking) with textile materials and structures? Research in HCI and other domains has often characterized e-textiles integration as creating a ``hybrid'' or ``composite'' system composed of some parts from textiles and others from electronics, carrying these metaphors into resulting design tools for the field. Yet these descriptions often do not capture the entangled histories of what we call ``textiles'' and ``computing'', nor the complex sociotechnical ecosystem that includes embodied knowledge (``craft'') that translates across domains and unequal power dynamics stemming from textiles' exclusion from a ``high tech'' vernacular. As part of my dissertation, I contend that the shortcomings of these metaphors can propagate from the design and system feasibility stages of e-textiles development into end-use applications, such as personalized healthcare wearables, and ultimately impact the effectiveness of e-textile designs when attending to multifaceted sociotechnical factors such as human intimacy, cultural diversity, and sustainable living. Therefore, I draw from work in critical design discourses and feminist philosophy to advance coproduction as an alternative metaphor to ``hybrid''-focused ones, first proposed by Devendorf and Rosner for HCI design \todo{[comps 34]} and which I explore for smart textiles design, seeking to account for the rich histories and social subjectivities that one might encounter in both textiles and electronics in an attempt to integrate the two. Simply put, smart textiles has not adequately engaged with textiles; I assert that coproduction, as a generative metaphor, offers better engagement with the technology's existing practices.

% Craft is more than what I make; it is also why I make I had been firmly aligned in STEM for so many years because My work has long been invested in technology for the sake of creating better futures, operating out of a sense of duty and evaluating how ``beneficial'' my impact would be, based on the end-use applications that could come out of the work. This was how I found myself drawn towards the applied physics of computing systems for my undergraduate studies, because surely advancing such a fundamental technology would have the largest possible impact. (Obviously, my thinking has changed since I was a teenager.) However, I did not choose to pursue e-textiles research through this sort of future-oriented rationalization.

% e-textiles and smart textiles can largely be used interchangeably

% ``E-textiles'' and ``smart textiles'' have both been used to refer to this emergent domain of technology that integrates textiles with digital electronics. 

% As the contested name suggests, this interdisciplinary field attracts a multiplicity of perspectives and work practices within a shared space. 

% Even as a growing field, E/ST already faces issues grounded in contemporary sociopolitical problems. 

% Specifically, we will focus on E/ST's concerns about sustainability. 

% As researchers in human-computer interaction (HCI), fashion, and other disciplines have noted, E/ST products could potentially contribute to both electronic waste ("e-waste") and textile waste, two of the largest global waste streams
% % SOURCES
% \cite{sandin_environmental_2018, forti_global_2020}.

% In smart textiles, practitioners combine textiles and digital electronics technologies to create future ``smart'' objects wherever textiles currently exist in the built environment: wearable devices, garments, home furnishings, vehicle seating, medical settings, and many more. 

% The field is projected to develop into a multi-billion USD industry in the coming years, owing to the wide variety of applications to healthcare, smart vehicle upholstery, and even next-generation spacesuits. 

% For designers of the emergent technology, smart textiles design tools enable them to realize these future applications -- e.g. visualizing circuit paths in a woven fabric. 

% Yet, there are very few (if any) products on the global market and fewer still that have become everyday consumer devices -- many prototypes struggle with durability, usability, and manufacturability. 

% Furthermore, like many technologies, there are many open problems surrounding the sustainability, inclusivity, and accessibility of future smart textiles. 

\section{Coproduction: the ``Hybrid'' Nature of E-Textiles}

In the development of emergent technologies such as e-textiles, the descriptors ``hybrid'' and ``novel'' are often used for the designs, components, and systems produced. For instance, one term that largely overlaps with ``smart textiles'' and ``e-textiles'' is ``flexible hybrid electronics" \cite{schwartz_flexible_2017} to emphasize the novelty of integrating digital electronics with flexible substrates (e.g. textiles) for contemporary consumer devices. In HCI design research, ``hybrid craft'' is a term often used to describe projects involving craft and computation (e.g. \cite{buechley_crafting_2012,liu_decomposition_2019,tsaknaki_articulating_2017}), many leveraging parametric design (e.g. \cite{magrisso_digital_2018,efrat_hybrid_2016}), as well as representing ``retooling'' as discussed in the previous section. As Devendorf and Rosner explore in ``Beyond Hybrids'' \cite{devendorf_beyond_2017}, these ``hybrid'' discourses tend to treat their practices as ``new'' and distinct from their precedents, while other design metaphors such as ``intra-action'' or ``coproduction'' can surface other dimensions for a design practice, such as cultural contexts. They first presented craft coproduction as an alternative to ``hybrid'' metaphors, building off Haraway's use of coproductions to describe the mutual shaping of categories and boundaries (e.g. software/hardware, human/machine), they argue for technologies that can explore, rather than resolve, intersections between things we tend to see as ``different.'' These generative metaphors surface certain design possibilities and bring perspectives from other scholarship to a practice. Namely, coproduction in HCI can offer a greater sense of different agencies, both human and non-human, and how they bring continuing legacies into the process of designing technology than ``hybrid'' discourses which can focus on a novelty that disrupts past designs, end-user goals, and features in a technology.

Synthesizing the definition from HCI craft with other usages in philosophy and society, technology, and science (STS or ST\&S) studies, I will specify that coproduction in my research means:

% \noindent\begin{minipage}{\linewidth}
  \begin{quote}
    
  \texttt{COPRODUCTION IN E-TEXTILES}

  The dynamic by which an emergent, interdisciplinary field of technology such as e-textiles is shaped by many agents in dialogue, including human designers and researchers, preceding technologies and epistemologies from textiles, electronics, etc., and sociocultural contexts. Coproduction acknowledges the agencies of non-humans in a e-textiles design practice, such as looms and other fabrication equipment, the materials (yarns) used and their differing properties (e.g. conductivity vs. softness), the sources of these materials, and long legacies from textile histories.
  \end{quote}
% \end{minipage}

\noindent{}In addition to the aforementioned sources, I draw from Sheila Jasanoff's elaboration of ``coproduction'' (which she styles as ``co-production") for contemporary STS, attempting to survey the many related threads opened up by previous scholars such as Latour, Pickering, Haraway, Foucault, etc. Coproduction is the realization that how humans make sense of the world is by two broad ``ordering'' schemes: ``the ordering of nature through knowledge and technology and the ordering of society through power and culture'' \cite{jasanoff_states_2010}. Jasanoff explains how two constructions mutually shape one another, and cannot be separated or given primacy that one is deterministic of the other. Furthermore, she includes ``hybrids'' as something that can be accounted for and better explained in context with a ``coproduction'' landscape. For instance, Latour and Woolgar write in Laboratory Life \cite{latour_laboratory_2013} on how scientific knowledge is socially constructed, and thus inseparable from social, economic, and political contexts. Haraway's work, which Devendorf and Rosner directly reference, offers a feminist take on ``co-production'' in mutual shaping of categories and boundaries, considering the (unequal) power dynamics between these actors \cite{haraway_staying_2016}. \revision{I ultimately align myself with an amalgam of these approaches to coproduction because of my emphasis on craft. As a craftsperson, coproduction describes to complex network of tools, materials, and techniques created by human and more-than-human agents alike, a dynamic which I must treat respectfully with my own hands.}

\section{Synopsis}

The rest of my dissertation will develop the synthesis of these themes and key concepts: \textbf{\textit{retooling e-textiles for coproduction}}. It is the design orientation that I have adopted as a designer in order to envision a future technology that respects textiles practices and meaningfully engages with the entailed baggage, namely how today's computing technology is both rooted in various textile traditions and metaphors, but is also enabled by global histories of colonization and industrialization that specifically have exploited the textiles practices, other traditional knowledges, and lands of Indigenous peoples \cite{nakamura_indigenous_2014}. I choose the term \textit{design orientation} as the metaphor that captures how both ``retooling'' and ``coproduction'' are concepts that allow me to maintain sustainability as a motivating ideal for my e-textiles design practice. Similar to how a compass, map, and other instruments (mentally internalized or externally constructed) allow sailors to navigate oceans towards distant goals, a design orientation provides a sense of direction and guidelines for course correction towards my envisioned goal of sustainable e-textiles. 

The next chapter will give further background on how craft relates to \revision{digital fabrication}, textiles, and social justice discourses at play in e-textiles. Following that, I will present three studies which I carried out during my PhD studies, \revision{but in four chapters. One mixed-methods study has been split into two distinct outcomes, thus creating Chapters \ref{ch_e-textiles} and \ref{ch_speculations}.}
% he research that I have already undertaken with collaborators to explore both coproduction and retooling as recurring themes in a design practice. Starting with my first-year experience as part of developing AdaCAD, a smart textiles CAD tool for woven circuitry, I recount how coproduction was first formulated among my colleagues and I to describe smart textiles. 
Starting with one phase of the mixed-methods study (Ch. \ref{ch_e-textiles}), I present the results of surveying e-textiles practitioners to deconstruct the language that shapes the field's collective identity and disciplinary values. Shifting to some physical deconstruction, Chapter \ref{ch_unfabricate} recounts my work on Unfabricate, an exploration in designing e-textiles for disassembly and reuse. I frame disassembly as a coproduction and as the central design tactic which motivated extensive retooling throughout my process. Chapter \ref{ch_loom-pedals} will present the process of crafting a customizable set of pedals for the TC2 digital Jacquard loom, the loom with which I have been weaving and retooling throughout my PhD research. We finally return to the first study \ref{ch_speculations}, a collaborative social research study conducted with a flexible electronics start-up to better understand our overlapping communities of practice. Together, these studies represent case studies in retooling e-textiles for coproduction, ranging from electronic hardware to discursive tactics.
% I describe how the social implications of discussing coproduction along with other perspectives on smart textiles led me to explicitly center retooling, in the design justice sense, as part of my practice. 

To close, I will reflect on how craft was the thread which strung together the entirety of my PhD research. As a result of these three keystone projects, along with tinkering on smaller projects on the side as I always do, I will summarize my understanding of retooling e-textiles for coproduction as it stands, and the open questions still hanging.

% Finally, in the Proposed Work section, I will identify and outline a project that addresses some of the unresolved questions: creating a  This work will create a design tool that is situated within my smart textiles ecosystem of practice, codifying ``retooling'' and ``coproduction'' as they had evolved in my local community of practice and enabling others in the broader experimental weaving community to collaboratively retool their smart textiles design practices.


% to my influences from this justice-oriented design practice in my Literature Review, as well as other work that motivates retooling and coproduction in the smart textiles domain. 

% Tools influence how designers perceive and use materials in their practice, as well as propagate values through the process by supporting or discouraging (intentionally or not) certain actions by designers. In the case of smart textiles, including the broader history of textiles, design choices in tools have not only influenced which techniques are easier for prototyping and which materials are more desirable to use, but they have also transmitted values from other established technologies -- values such as the product design paradigm's ``manufacturability'' and ``user-friendliness". Critical scholars from fields such as design, sociology, history, and science, technology, and society (STS) have investigated how such values have reinforced social inequities and accelerated global problems such as climate change [14,66,104], as successful tools build user bases and become components of infrastructure. My belief is that retooling smart textiles to center coproduction will enable the field to interrogate its potential impact out of respect for textiles, which may reduce future waste in manufacturing and product development and mindfully attend to local materials, community needs, and diverse design practices.


% To summarize, smart textiles (also referred to as ``e-textiles") is an emergent technology that integrates electronic functionality (sensing, actuating, wireless networking) with textile materials and structures. My dissertation develops the concept of ``coproduction'' as an alternative model for situating smart textiles design within multiple entangled technological domains. I use the concept of ``retooling'' from design justice to align my focus on design tools and HCI with values-driven design discourses around sustainability and intersectional justice. This dissertation uses smart textiles as a case study for the effects of coproduction and retooling on a design domain's values.

% As part of that vision, my research engages with ongoing questions concerning the sustainability of future technologies, which I see as part of a broader, intersectional movement for justice. While it is not in the scope of my PhD to directly address what sustainable e-textiles will be, or to evaluate the impact of my work on sustainability efforts in the field, my focus on domain-specific design tools will enable me to continue engaging with sustainability in future stages of my career. I also acknowledge the agenda behind my choice of the word ``retooling'', as I use the term in alignment with its definition from design justice. 

% I will give a more specific example of coproduction in [AdaCAD section of background]

% Throughout this dissertation, by reviewing the studies undertaken during my PhD studies, I will explore how craft has been the thread that stitches together 
% manifesting in two distinct themes, ``retooling'' and ``coproduction". Both terms require some introduction, at least in how they are used in design research.
